\section{Фундаментальные группы}
\begin{Intro}
    На самом деле гомотопические классы $\pi(X,Y)$ --- это очень широкое понятие. С ним тяжело работать, так как не всегда понятно, что это за пространства. Одним из стандартных сужений является идея рассматривать гомотопические классы только отображений $C\left( (I^n, \partial I^n), (X, x_0)\right)$ (или то же самое в виде $C\left( (S^n, \{\mathrm{pt}\}), (X, x_0)\right)$) и соответствующие гомотопические классы, называемые  гомотопическими группами $\pi_n(X, x_0)$  --- они действительно группы  и именно это делает их крутыми. Но вместе с высокими размерностями приходят большие сложности, поэтому мы лишь остановимся на простом случае $\pi_1$.
\end{Intro}
\subsection{Определение и свойства}
\begin{Note}
    Кстати, также можно оценить симпатичность $\pi_0(X, x_0)$. 

    Как мы видели, это просто классы отображений $C((S^0, \{\mathrm{pt}\}), (X, x_0))$, то есть отображений из двух точек с выделенной одной точкой. Нетрудно понять, что это просто то же самое, что разделение на линейные компоненты связности.
\end{Note}

Введем тогда еще раз строго определение для первой гомотопической группы, которую приличные люди еще называют фундаментальной группы. 
\begin{Def}
   Фундаментальной группой пространства $(X, x_0)$ с выделенной точкой назовем 
   \[
       \pi_1(X, x_0) = \pi((S^1, \{\mathrm{pt}\}), (X, x_0)).
   \] 
\end{Def}

Для того, чтобы понять это определение не только на уровне страшной формулы, введем достаточно очевидные определения (одно из них мы на самом деле уже формулировали для линейной связности).
\begin{Def}[Путь, петля]\mbox{}
    \begin{itemize}
        \item Путем, начинающимся в точке $x$ и заканчивающемся в точке $y$ пространства  $\X$, назовем непрерывное отображение $\gamma \st I=[0,1] \to \X$, что $\gamma(0) = x$ и $\gamma(1) = y$.
        \item Петлей в точке $x_0$ назовем путь, начинающийся и заканчивающийся в одной точке $x_0$.
        \item Сложением петель $\gamma_1$ и $\gamma_2$ в одной точке назовем последовательное прохождение этих петель, то есть
            \[
                (\gamma_1 \circ \gamma_2) (t)= \begin{cases}
                \gamma_1(2t), \quad t \in [0, \frac{1}{2}],\\
                \gamma_2(2t-1), \quad t\in [\frac{1}{2}, 1].
            \end{cases}
            \] 
    \end{itemize}
\end{Def}

\begin{Task}
    Ради аккуратности изложения стоит что-то сказать об адекватности введенной операции над петлями. В действительности, оставляем в качестве маленького упражнения убедиться, что сумма двух петель (или, как модно говорить, конкатенация) является петлей.
\end{Task}

Теперь можно что-то сказать про то, что же такое страшное было дано в определении фундаментальной группы.

\begin{Note}
    Заметим, что в определении $\pi_1(X, x_0) = \pi((S^1, \{\mathrm{pt}\}),(X, x_0))$ вся суть содержится в пространстве отображений из окружности $S^1$ с выделенной точкой в наше пространство. Иначе говоря, это как раз всевозможные петли в точке $x_0$. 

    И оказывается, что в $\pi_1$ лежат гомотопические классы петель в точке $x_0$.

    А как мы помним, гомотопические классы интуитивно можно понимать как наличие путей между отображениями. То есть мы просто рассматриваем петли с точностью до возможности непрерывно перевести одну из них в другую.

    Очень грубо и в то же время элегантно это можно показать, взяв в руки лассо --- которое удивительным образом является петлей --- и набрасывать его на наше пространство. Путем затягивания лассо мы сможем перейти от одного состояния лассо (одной петли) к другому (другой петли) --- это гомотопия. Но может так оказаться, что одну петлю в другую не стянуть. Это легко увидеть на торе: рассмотрим петлю, которая делает оборот вокруг всего тора, и любую простую небольшую петлю (Рис. \ref{fig:torus}). Можно заметить, что эти две петли разные. Более того, первой петлей лассо после затягивания поймает наш тор, а  вторая петля-лассо просто затянется в узелочек.

    Именно из такого соображения ловли чего-то после затягивания лассо часто говорят, что фундаментальная группа отображает количество дырок, которые есть в пространстве (где под дырками и понимается то, что ловится в наше лассо).
\end{Note}

Вообще название этого понятия "фундаментальная группа" несет в себе очевидную теорему, которую нужно проверить.
\begin{Th}
    $\pi_1(X,x_0)$ вместе с операцией сложения петель является группой.
\end{Th}
\begin{Proof}
    Необходимо доказать, что 
    \begin{enumerate}
        \item операция сложения переносится на классы адекватно,
        \item есть ассоциативность сложения,
        \item есть нейтральная по сложению петля (подсказка: эта та, в которой мы никуда не уходили из $x_0$),
        \item есть обратная петля (подсказка: просто пройти обратно по той же петле).
    \end{enumerate}

    И мне лень сейчас это писать TODO
\end{Proof}

Вообще нахождение фундаментальной группы оказывается совершенно непростой задачей и вокруг этого строится огромное количество теорем. Пока приведем простой пример фундаментальной группы, который даже сейчас можем решить.

\begin{Ex}
    В выпуклом подмножестве $\mathbb{R}^n$ фундаментальная группа для любой точки есть  $\{e\}$, где $e$ --- петля, никуда не выходящая из точки.

    Нетрудно это показать, построив 
\end{Ex}



\subsection{Функториальность $\pi_1$ }
\subsection{Односвязность}

