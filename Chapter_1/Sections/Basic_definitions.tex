\section{Основные определения}
\subsection{Топология и топологическое пространство}
\begin{Def}[Топология]
    Рассмотрим произвольное множество $X$. Множество его подмножеств $\Omega$ называется топологией, если выполнен следующий набор свойств:
    \begin{enumerate}
        \item $\varnothing \in \Omega$, $X \in \Omega$;
        \item $\forall U, V \in \Omega  \implies U \cap V \in \Omega$;
        \item $\forall \alpha \in \mathcal{A} \; U_\alpha \in \Omega \implies \bigcup\limits_{\alpha\in\mathcal{A}} U_\alpha \in \Omega $.
    \end{enumerate}
\end{Def}

Расшифровывая данное определение, можно просто запомнить, что пустое и всё множество лежат в топологии (1-ое свойство), конечное пересечение множеств топологии лежит в топологии (2-ое свойство) и любое объединение множеств топологии лежит в топологии (3-ье свойство).
\begin{Note}
    Такое определение появилось неслучайно. Дело в том, что топология как наука создавалась достаточно поздно (в истории развития математики). Поэтому в других частях математики (особенно в матанализе) уже были построены некоторые идеи, которые при развитии топологии хотелось оставить действующими и в ней ради целостности математики.

    Скорее всего, именно поэтому лишь конечное пересечение лежит в топологии. Ведь в матанализе нетрудно найти пример, в котором достаточно хороший набор подможеств -- интервалы -- не содержит в себе какое-то бесконечное пересечение. 
\end{Note}
\begin{Task}
    Попробуйте самостоятельно подобрать такой бесконечный набор интервалов, пересечение которого не будет являться интервалом.
\end{Task}

Понятно, что топология $\Omega$ не существует отдельно от множества $\X$. Поэтому правильнее будет рассматривать именно пару множество--топология, которая образует пространство. 

\begin{Def}
    [Топологическое пространство]
    Пара \topX множества $X$ с введенной топологией $\Omega_X$ называется топологическое пространство.
\end{Def}
\begin{Note}
    Так как большая часть последующих размышлений посвящена топологическим пространствам, то часто в дальнейшем мы будем опускать пару множество-топология и ограничимся лишь чуть более жирным написанием исходного множества -- $\X$.

    То есть под $\X$ стоит понимать как само множества, так и топологическое пространство \topX.
\end{Note}

\begin{Ex}
    Приведем пример самых наивных топологий:
    \begin{enumerate}
        \item $(X, \{ \varnothing, X \})$ - антидискретная топология, состоящая из двух элементов
        \item $(X, 2^X)$ - дискретная топология, из всех подмножества множества $X$
    \end{enumerate}
\end{Ex}

\begin{Ex}
    Рассмотрим пример, так называемой, стандартной топологии, постоянно использующейся в одномерном  математическом анализе.
    Пусть в качестве множества  $X$ будет числовая прямая $\mathbb{R}$, а в качестве топологии $\Omega_{st}$ будет пустое множество и всевозможные объединения интервалов. 
    Коротко это можно записать $$\left( \mathbb{R}, \, \left\{ \cup(a,b) \st a,b \in \mathbb{R} \right\} \right)$$.
\end{Ex}
\begin{Task}
    Проверьте свойства топологии из определения для предложенных примеров.
\end{Task}

На самом деле стандартную топологию можно также задать и для $\mathbb{R}^n$. Об этом поговорим позже в пункте про индуцированные метрикой топологии.

\

Следующие определения просто вводят новые названия уже существующим объектам. Тем не менее, это необходимо из-за постоянного обращения к этим объектам в будущем.

\begin{Def}
    [Открытые и замкнутые множества]
    Пусть дано некоторое топологическое пространство \topX.
    \begin{enumerate}
        \item Элементы топологии $\Omega_X$  будем называть открытыми множествами,
        \item Множества $X\setminus U$, где $U\in \Omega_X$, будем называть замкнутыми множествами.
    \end{enumerate}
\end{Def}
\begin{Note}
   Заметьте, что открытые и замкнутые множества (как и в русском языке слова открытое и замкнутое) вовсе не являются противоположными.
   Так, множество может быть 
   \begin{itemize}
       \item \textit{и открытым, и замкнутым}, как пустое и $X$ (но не всегда только они),
       \item \textit{открытым, но не замкнутым}, как интервал в $(\mathbb{R}, \Omega_{st})$,
       \item \textit{замкнутым, но не открытым}, как отрезок в $(\mathbb{R}, \Omega_{st})$,
       \item \textit{ни замкнутым, ни открытым}, как полуинтервал в $(\mathbb{R}, \Omega_{st})$.
   \end{itemize}
\end{Note}

\begin{Note}
    На самом деле открытые и замкнутые множества являются даже весьма схожими объектами. Так, топологию можно определять через замкнутые множества. Для этого нужно немного модернизировать определение:

    Рассмотрим совокупность $\tilde{\Omega}$ подмножеств множества $X$, для которой 
    \begin{enumerate}
        \item $\varnothing \in \tilde\Omega$, $X \in \tilde\Omega$ --- эти условия никак не меняются (ведь мы знаем, что эти множества одновременно открыты и замкнуты),
        \item $\forall F, G \in \tilde\Omega \implies F \cup G \in \tilde\Omega$,
        \item $\forall \alpha \in \mathcal{A} \; F_\alpha \in \tilde\Omega \implies \bigcap\limits_{\alpha\in\mathcal{A}}F_\alpha \in \tilde\Omega $.
    \end{enumerate}
    Тогда $\tilde\Omega$ описывает всевозможные замкнутые множества $X$, а топологией можно назвать $\Omega = \left\{ A\subset X \st X \setminus A \in \tilde\Omega \right\}$.
\end{Note}

\begin{Note}
    В дальнейшем, мы будем использовать пару значков, чтобы сделать записи в доказательствах более удобными. Договоримся:
    \begin{itemize}
        \item $U \open X$ будет означать, что $U$ открыто в $\X$,
        \item $F \closed X$ будет означать, что $F$ замкнуто в $\X$
    \end{itemize}
\end{Note}

\subsection{База топологии}
Мы примерно разобрались с тем, что такое топология. Однако у нас до сих пор нет никакого способа описания топологического пространства, не описывая всевозможные открытые множества. По этой причине предлагается следующий объект, позволяющий описать некоторую часть топологии, которой будет достаточно для восстановления всей структуры.
\begin{Def}
    [База топологии]
    Назовем совокупность $\mathbb{B}$ открытых множеств $\X$ базой топологии \topX, если всякое непустое открытое множество этой топологии можно представить в виде объединения элементов этой совокупности.
\end{Def}
\begin{Ex}
    Так в качестве базы стандартной топологии можно рассмотреть множество всевозможных интервалов с вещественными концами. 
    \[
        \mathbb{B}_{st} = \left\{ (a, b) \st a,b \in \mathbb{R} \right\}.
    \] 

    Аналогично, можно рассмотреть только интервалы с рациональными концами. Такое множество тоже будет базой.
\end{Ex}

\begin{Task}
    Подумайте, могут ли различные топологические структуры иметь базу?
\end{Task}

Можно заметить, что какие-то базы могут порождать одни и те же топологии. Для различия баз определим, когда можно говорить про базы, как про одинаковые объекты.
\begin{Def}
    [Эквивалентные базы]
    Базы называются эквивалентными, если они порождают одну и ту же топологию.
\end{Def}

   \subsection{Метрика и ее связь с топологией}
   Наверняка вы уже встречались ранее с понятием метрики на различных курсах по математике (а может и не только). Однако для строгости изложения и в целях напоминания приведем некоторые отрывки из теории метрических пространств.
   \begin{Def}
       [Метрика]
       Функция $\rho\st X \times X \to \mathbb{R}$ называется метрикой, если выполнено
       \begin{enumerate}
           \item $\rho(x,y) = 0 \Leftrightarrow x = y$,
           \item $\forall x,y \; \rho(x,y) = \rho(y,x)$,
           \item $\forall x,y,z \; \rho(x,y) + \rho(y,z) \geqslant \rho(x,z)$.
       \end{enumerate}
   \end{Def}
   \begin{Def}
       [Метрическое пространство]
       Множество $X$ и метрика $\rho$ на нём образуют метрическое пространство $(X, \rho)$.
   \end{Def}

   \begin{Def}
       [Шары и сферы]
       В метрическом пространстве $(X, \rho)$ для точки $a\in X$ и произвольного положительного вещественного числа $r\in \mathbb{R}_+$ вводятся понятия:
       \begin{enumerate}
           \item Открытого шара $B_r(a)$
               \[
                   B_r(a) = \left\{ x\in X \st \rho(a, x) < r \right\},
               \] 
           \item Замкнутого шара $\overline{B}_r(a)$ 
               \[
                   \overline{B}_r(a) = \left\{ x\in X \st \rho(a, x) \leqslant r \right \},
               \] 
           \item Сферы $S_r(a)$ 
               \[
                   S_r(a) = \left \{ x \in X \st \rho (a, x) = r \right \}.
               \] 
       \end{enumerate}
   \end{Def}

   \begin{Note}
       Важно понимать, что термины <<шар>>, <<сфера>> не всегда передают реальную форму шаров и сфер. 

       Так, как бы странно это не звучало, при разных введенных метриках в $\mathbb{R}^2$ шар может оказаться квадратом, ромбом (при этом в самом привычном нам случае он в действительности окажется шаром, только двумерным, то есть кругом).
   \end{Note}
   \begin{Task}
       Найдите примеры метрик в $\mathbb{R}^2$ с разными формами шаров. 
   \end{Task}

   Оказывается, что в метрическом пространстве всегда есть одна понятная топология, которую мы будем называть метрической топологией.

   \begin{Def}
       [Метрическая топология]
       Множество всевозможных шаров некоторого метрического пространства является базой некоторой топологии. Такая топология называется порожденной метрикой топологией или просто метрической топологией.
   \end{Def}
   \begin{Ex}
       Простейшим примером такой топологии является стандартная топология. Только теперь мы можем описать ее не только для одномерного случая.

       Стандартной топологией для $\mathbb{R}^n$ будем называть топологию, индуцированную евклидовой метрикой.
   \end{Ex}

    В метрической топологии можно немного по-другому рассматривать открытость множества. Именно таким образом обычно обходят страшную \textit{топологию} в курсах математического анализа.

   \begin{Prop}
       В порожденной метрикой топологии множество является открытым тогда и только тогда, когда оно содержит каждую свою точку вместе с некоторым шаром, центром которого она является.
   \end{Prop}
   \begin{Proof}
       $\Rightarrow$ Пусть множество $A$ открыто. Тогда оно является объединением некоторых шаров  $A = \bigcup\limits_{\alpha\in\mathcal{A}}B_{r_\alpha}(y_\alpha)$.   

       Для произвольной точки $x \in A$ найдем тот из этих шаров, которому она принадлежит. Пусть это просто $B_r(y)$. Тогда шар $B_{r-\rho(x,y)} (x)$, где $\rho$ -- метрика, является искомым.

       \

       \noindent$\Leftarrow$ Рассмотрим для каждой точки $x$ шар $B_r(x)$ из условия. Тогда $\bigcup\limits_{x\in A} B_r(x) = A$ и при этом, так как все шары открыты, то это объединение открытых множеств --- а значит открытое.
   \end{Proof}

   \begin{Task}
       Проверьте, что замкнутые шары являются замкнутыми множествами, а открытые шары -- открытыми множествами.
   \end{Task}

   Некоторые топологические пространства могут быть порождены метрикой, даже если мы этого не подозреваем (или просто определяем без отсылок к ней). Однако такие топологии образуют группу, которая имеет свои преимущества и упрощения перед остальными топологиями.

   \begin{Def}
       [Метризуемые пространства]
       Топологическое пространство называется метризуемым, если его топологическая структура порождается некоторой метрикой.
   \end{Def}

   \begin{Note}
       Отметим, что далеко не все топологии являются метризуемыми. Простейшим (но не самым показательным) примером неметризуемого пространства является антидискретная топология, состоящая из более чем одной точки.
   \end{Note}
   \subsection{Топология на подпространстве}
   Можно пробовать строить топологию на основе уже имеющихся. Простейшие варианты мы рассмотрим в ближайших двух пунктах, а более конструктивные способы будут представлены в 7 и 8 параграфах. 
   %TODO links
   \begin{Def}
       [Индуцированная топология]
   Рассмотрим некоторое подмножество $A\subset X$ пространства \topX. Совокупность  $\Omega_A = \left\{ A \cap U \st U \in \Omega_X \right\}$ является топологией в множестве $A$. Такую топологию называют индуцированной в $A$ топологией.
   \end{Def}
    \begin{Task}
        Проверьте, что индуцированная топология действительно является топологией по определению.
    \end{Task}

    \begin{Prop}
        Множество $F$ является замкнутым в подпространстве $A\subset X$ тогда и только тогда, когда $F = A \cap E$, где $E$ -- замкнуто в $X$.
    \end{Prop}
    \begin{Proof}
        $\Rightarrow$ Пусть $F \closed A$. Тогда $A\setminus F \open A$ и это  множество представимо в виде $A \setminus F = A \cap U = A \cap (X \setminus E) = (A \cap X) \setminus (A \cap E) = A \setminus (A\cap E)$, где $U \open X$, $E \closed X$. Избавляясь с двух сторон от $A$, получаем искомое.

        \

        \noindent$\Leftarrow$ Пусть $F = A\cap E$. Положим $U = X \setminus E \open X$. Тогда $A \cap U = A \cap (X\setminus E) = (A\cap X) \setminus (A \cap E) = A \setminus F$. Но $A \cap U \open A$, а значит и $A \setminus F$. А тогда $F \closed A$.
    \end{Proof}
    
    \begin{Note}
        Заметим, что множества, являющиеся открытыми в подпространстве вовсе не всегда открыты в объемлющем пространстве.

        Так, рассмотрим стандартную топологию на $\mathbb{R}$ как индуцированную из топологии на $\mathbb{R}^2$. Единственным открытым множеством из $\mathbb{R}$, которое открыто в $\mathbb{R}^2$ будет пустое множество. 

        Такое свойство часто называют относительностью открытости.
    \end{Note}

 Однако иногда все же открытость в подпространстве равносильна открытости в объемлющем пространстве. Рассмотрим это в следующем предложении.

    \begin{Prop}
        Открытые множества открытого подпространства являются открытыми и во всем пространстве.
        \[
        A \open \X  \implies \forall U \open A \implies U \open \X.
        \] 
        или еще проще
        \[
        A \open \X \implies \Omega_A \subset \Omega_X.
        \] 
    \end{Prop}
    \begin{Proof}
        Пусть $U\open A$. Тогда по определению $U = A \cap V$, где $V \open X$. И получается, что, так как $A \open X$, то $U$ есть объединение двух открытых в $X$. А значит оно само открыто и $U \in \Omega_X$. 
    \end{Proof}

    \subsection{Топология произведения}
    Вспомните идею при построении декартова произведения множеств. Фактически мы предъявляем упорядоченную пару. Аналогично можно построить топологическое пространство по двум (или нескольким) топологиям. 
    \begin{Def}
        [Топология произведения]
        Рассмотрим два топологических пространства \topX и \topY. Тогда на декартовом произведении  $X \times Y$ можно рассмотреть топологию, порожденную базой
         \[
             \mathbb{B} = \{ U \times V \st U \open X, \; V \open Y\}
        \] 
    \end{Def}
    \begin{Ex}
        Заметим, что стандартная топология на $\mathbb{R}^2$, к примеру, совпадает с топологией произведения стандартных топологий на $\mathbb{R}$. 
    \end{Ex}

    \subsection{Расположение точек относительно множества}
    Пока у нас нет никакого способа определять открытые множества без разложения в объединение других открытых. Как один из таких вариантов, можно рассматривать различные точки и окрестности вокруг них. На основе этих окрестностей можно определить разные классы точек пространства. 
    \begin{Def}
        Пусть \topX --- топологическое пространство,  $A \subset X$. Точка $b\in A$ называется:
        \begin{enumerate}
            \item внутренней для множества $A$, если есть окрестность этой точки, полностью лежащая в  $A$
                \[
                    b \text{ -- внутренняя, если } \exists U \ni b \st U \subset A.
                \] 
            \item внешней для множества $A$, если есть окрестность этой точки, не пересекающаяся с $A$
                \[
                    b \text{ -- внешняя, если } \exists U \ni b \st U \cap A  = \emptyset.
                \] 
            \item граничной для множества $A$, если любая окрестность этой точки, пересекается с $A$ и с $X \setminus A$
                \[
                    b \text{ -- граничная, если } \forall U \ni b \implies U \cap A  \neq \varnothing \text{ и } U \cap (X \setminus A) \neq \varnothing.
                \] 
        \end{enumerate}
    \end{Def}
    Понятно, что все внутренние точки образуют некоторое множество. Интересно, что это множество можно задавать и другим способом. 
    \begin{Def}
        Пусть \topX --- топологическое пространство,  $A \subset X$. Внутренностью $\mathrm{Int}(A)$ множества $A$ называется:
          \begin{enumerate}
              \item множество его внутренних точек,
              \item объединение всех открытых множеств, лежащих в $A$. 
          \end{enumerate} 
    \end{Def}
    \begin{Prop}
        Определения внутренности эквивалентны. 
    \end{Prop}
    \begin{Proof}
        $\Rightarrow$ Если точка лежит вместе с некоторой окрестностью, а окрестность в свою очередь лежит в $A$, то она лежит и в объединении всех открытых множеств, лежащих в A
        
        \

        \noindent $\Leftarrow$ Предположим, что точка $x$ лежит в $A$ с некоторой окрестностью, но не входит в объединение всех открытых множеств лежащих в $A$. Получаем явное противоречие. 
    \end{Proof}

    Так как мы увидели, что внутренность является открытым множеством, то появляется способ определения открытых множеств.
    \begin{Prop}
        $A \open X$ тогда и только тогда, когда $\mathrm{Int} (A) = A$.
    \end{Prop}
    \begin{Proof}
        $\Rightarrow$ Обозначим $\mathcal{U} = \{ U \in \Omega_X \st U\subset A\}$. Так как множество открыто, то $\mathrm{Int} (A) = \bigcup\limits_{U \in \mathcal{U}}{U} = A$, так как $A$ само одно из этих множеств в объединении.
        
        \

        \noindent $\Leftarrow$ Очевидно, так как $\mathrm{Int} (A)$ открыто как объединение открытых.
    \end{Proof}

    Аналогично можно ввести понятие внешности. 

    Говоря про классификацию точек, можно рассмотреть немного другой подход. 
    \begin{Def}
        Пусть \topX --- топологическое пространство,  $A \subset X$. Точка $b\in A$ называется:
        \begin{enumerate}
            \item точкой прикосновения для $A$, если любая окрестность пересекается с $A$
                \[
                \forall U \ni b \implies U \cap A \neq \emptyset.
                \] 
            \item предельной точкой, если любая проколотая окрестность пересекается с $A$
                \[
                    \forall U \ni b \implies U \cap (A \setminus \{a\}) \neq \emptyset.
                \] 
        \end{enumerate}
    \end{Def}
    С такими точками тоже можно ввести некие множества.
    \begin{Def}
        Замыканием $\mathrm{Cl}(A)$ множества $A \subset X$ называется множество его точек прикосновения.
    \end{Def}
    \begin{Prop}
        Замыкание равно пересечению всех замкнутых множеств, содержащих $A$.
    \end{Prop}

    \begin{Prop}
        $A \closed X$ тогда и только тогда, когда $\mathrm{Cl}(A) = A$.
    \end{Prop}
    \begin{Task}
        Докажите данные утверждения аналогично утверждению про внутренность.
    \end{Task}

    Мы знаем из курса матанализа, что множество рациональных чисел всюду плотно в множестве вещественных. Там это выражалось в смысле: между любыми двумя вещественными числами можно найти рациональное. Однако плотность можно вводить, используя замыкание, для любых пространств.
    \begin{Def}
        Пусть $A, B \subset X$. Говорят, что
        \begin{enumerate}
            \item $A$ плотно в $B$, если $B \subset \mathrm{Cl}(A)$
            \item $A$ всюду плотно в $X$, если $\mathrm{Cl}(A) = X$.
        \end{enumerate}
    \end{Def}
    \begin{Ex}
        Как уже говорилось, $\mathbb{Q}$ всюду плотно в $\mathbb{R}$. Также $\mathbb{I}$ всюду плотно в $\mathbb{R}$. 
    \end{Ex}

    \subsection{Последовательности}
    Как и в матанализе, можно рассматривать последовательности из точек пространства. Определение последовательности и предела совершенно не отличается от привычного. Однако оказывается, что, в отличие от стандартной топологии, не всегда предел единственный.
    \begin{Def}
        [Последовательность]
        Последовательностью в пространстве $X$ назовем отображение $q\st \mathbb{N} \to X$.
    \end{Def}
    \begin{Def}
        [Сходящаяся последовательность]
        Говорят, что последовательность $\{x_n\}_{n=1}^\infty$ сходится к $a \in X$, если \[
            \forall U(a) \open X \implies \exists N \in \mathbb{N} \st \forall n > N \implies x_n \in U(a).
        \] 
    \end{Def}
    \begin{Ex}
        Рассмотрим антидискретную топологию на $\mathbb{R}$ и последовательность $\{1\}_{n=1}^\infty$. Любое натуральное число является пределом такой последовательности.
    \end{Ex}
    Данный пример показывает, что не всегда предел единственный. Мы еще увидим далее, какого свойства будет достаточно для единственности предела. 
