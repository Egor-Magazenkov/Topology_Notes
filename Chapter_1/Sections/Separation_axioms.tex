\section{Аксиомы отделимости}
\begin{Intro}
    В данном параграфе речь пойдет о первом (для нас) топологическом инварианте, связанном с тем, как мы можем отделить какие-то множества (в простейших случаях точки) друг от друга.

    На самом деле, за время существования науки свойств, связанных с отделимостью, придумали достаточно много. Мы ограничимся наиболее важными для нас, но посмотреть их (возможно, неполный список) можно, к примеру, на \href{https://en.wikipedia.org/wiki/Separation_axiom#Main_definitions}{википедии}.
\end{Intro}

Забавно, что стандартные аксиомы называются <<T аксиомы>>. На самом деле корни у этого идут их специально придуманного немецкого слова Trennungsaxiom (разделение + аксиома). 

\subsection{Аксиома $T_0$, Колмогорова}
\begin{Def}
    [Аксиома $T_0$, Колмогорова]
    Топологическое пространство \topX удовлетворяет нулевой аксиоме отделимости (аксиоме Колмогорова или, проще говоря, является $T_0$ пространством), если для любых двух точек можно найти окрестность, содержащую одну точку и не содержащую другую.

    Переписывать данное определение символьно не очень удобно, но тоже можно
    \[
    \forall x, y \in \X \implies \exists U\open \X  \st  (x \in U \wedge y \notin U) \vee (y \in U \wedge x \notin U).
    \] 
\end{Def}

\begin{Ex}
    Практически любое привычное пространство является $T_0$.
\end{Ex}
При этом все-таки не совсем все пространства удовлетворяют аксиоме Колмогорова. В частности, в следующем примере показано, почему антидискретная топология не является таковой.
\begin{Ex}
    Антидискретная топология не является $T_0$ пространством.

    Действительно, в антидискретной топологии открытыми являются только пустое множество и само множество $X$. А значит это единственные окрестности, которые можно рассматривать. Но они всегда либо содержат обе точки (в случае всего множества), либо не содержат ни одной (в случае пустого множества). А значит определение не выполняется и такая топология не удовлетворяет аксиоме $T_0$.
\end{Ex}


\subsection{Аксиома $T_1$, Фреше}
\begin{Def}
    [Аксиома $T_1$, Фреше]
    Топологическое пространство \topX удовлетворяет первой аксиоме отделимости (аксиоме Фреше или, проще говоря, является $T_1$ пространством), если для каждой точки можно найти окрестность, не содержащую любую другую точку.

    Символьно можно записать это как
    \[
    \forall x\in \X \; \forall y\in \X \implies \exists U_x\open \X \st y \notin U.
    \] 
\end{Def}

Аксиомы отделимости постепенно будут добавлять какие-то свойства на пространство. Первым оказывается привычное нам свойство замкнутости точки.

\begin{Lem}
    Пространство является $T_1$-пространством тогда и только тогда, когда любое одноточечное множество является замкнутым. 
\end{Lem}
\begin{Proof}
    $\implies$ Рассмотрим произвольную точку $x\in X$. Для каждой точки $y_i \ne x$ найдем окрестность $U_{y_i}$ из определения $T_1$ пространства (то есть содержащую $y_i$ и не содержащую $x$ ). 

    Объединяя все такие окрестности, мы фактически пройдем по всем точкам кроме самой $x$, а значит
    \[
        \underset{\tiny \stackrel{y\in X}{y\ne x}}{\cup} U_{y} = X \setminus \{x\} \open \X.
    \] 
    А тогда $\{x\} \closed \X$ как дополнение к открытому.

    $\Longleftarrow$ Для любой точки пространства $x$ можем рассмотреть окрестность $U = X \setminus \{x\} \open \X$. Понятно, что эта окрестность отделяет любую другую точку от $x$.
\end{Proof}

\begin{Ex}
    Практически любое привычное пространство является $T_1$.
\end{Ex}

\begin{Ex}
    Так называемое связное двоеточие $\left(\{a,b\}, (\left\{ \emptyset, a, \{a,b\} \right\}\right)$ не является $T_1$ пространством.

    Нетрудно проверить, что дополнение к множеству $\{a\}$, равное $\{b\}$ не является открытым. А значит точка $\{a\}$ не является замкнутым множеством и все пространство не может быть $T_1$.
\end{Ex}

Следующая лемма (как и аналогичные ей далее) приведется без доказательства во многом, потому что она очевидна. Однако эта лемма показывает основную идею при построении новых аксиом отделимости. Грубо говоря, <<следующая по номеру аксиома наследует свойства предыдущего>>. Это как бы позволяет на практике перебирать их последовательно.
\begin{Lem}
    \[T_0 \subset T_1\]
\end{Lem}

\subsection{Аксиома $T_2$, Хаусдорфа}
\begin{Def}
    [Аксиома $T_2$, Хаусдорфа]
    Топологическое пространство \topX удовлетворяет второй аксиоме отделимости (аксиоме Хаусдорфа или, проще говоря, является $T_2$ пространством), если для любых двух точек можно найти непересекающиеся окрестности.

    Символьно можно записать это как
    \[
    \forall x,y \in \X  \implies \exists U_x, V_y\open \X \st U_x \cap V_y = \emptyset.
    \] 
\end{Def}

\begin{Ex}
    Метрическое пространство является Хаусдорфовым.

    Это нетрудно заметить, построив соответствующие окрестности радиусами, равными половине расстояния между точками.
\end{Ex}

\begin{Ex}
    Прямая Зариского $\left(\mathbb{R}, \left\{ \mathbb{R} \setminus F \st F \text{ -- конечное} \right\} \right)$ не является $T_2$ пространством.

    Понятно, что в такой топологии пересечение любых двух открытых множеств $U_x = \mathbb{R} \setminus F_1$ и $U_y = \mathbb{R} \setminus F_2$ (где $F_i$ -- конечные множества) есть 
   \[
       U_x \cap U_y = (\mathbb{R} \setminus F_1 ) \cap (\mathbb{R} \setminus F_2) = \mathbb{R} \setminus (F_1 \cup F_2) \open \mathbb{R}.
   \] 
   При этом $F_1 \cup F_2$ конечно, как объединение конечных, а значит $\mathbb{R} \setminus (F_1 \cup F_2)$ непусто. То есть нарушается условие Хаусдорфовости.
\end{Ex}

\begin{Lem}
    Пусть $f\st \X \to \Y$ -- непрерывное отображение, причем $\Y$ -- хаусдорфово. Тогда график этой функции $\Gamma_f = \left\{ (x, f(x) \st x\in \X \right\}$ замкнут в произведении $\X \times \Y$.
\end{Lem}
\begin{Proof}
    Рассмотрим произвольные точки $x \in X$ и $Y \ni y \ne f(x)$.

    В силу Хаусдорфовости $Y$ найдутся непересекающиеся окрестности $U, V$ точек $y$ и $f(x)$ соответственно. Но тогда из непрерывности отображения $ \inv{f}(V) \open \X$, то есть множество $\{(x,y) \st y \ne f(x)\}$ открыто в произведении. А значит дополнение к нему, то есть $\Gamma_f$ замкнуто. 
\end{Proof}

\begin{Cor}
     \topX -- Хаусдорфово тогда и только тогда, когда диагональ $\Delta_x = \left\{ (x, x) \st x \in \X \right\}$ замкнута в $\X \times \X$.
\end{Cor}
\begin{Proof}
    $\implies$ Возьмем в предыдущей лемме $f(x) = x$.

    $\Longleftarrow$ Пусть $X$ не хаусдорфово. Рассмотрим $H = X \times X \setminus \Delta_x$. Так как $\Delta_x \closed X \times X$, то $H \open X\times X$. Для любой точки $(x,y) \in H$ можно выбрать окрестность, полностью лежащую в $H$, как элемент базы (ведь открытое должно раскладываться в объединение элементов базы). А в базе топологии произведения элементы имеют вид $U \times V$. То есть $U \times V \subset H$ и $U \times V \cap \Delta_x = \emptyset$. А тогда и $U\cap V = \emptyset$. То есть для точек $x,y \in X$ нашли непересекающиеся окрестности, отделяющие их. Иначе говоря, $X$   -- хаусдорфово.
\end{Proof}

Более того, в хаусдорфовых пространствах уже предел последовательности оказывается единственным.
\begin{Prop}
    Если \topX является хаусдорфовым, то любая последовательность имеет ровно один предел.
\end{Prop}
\begin{Proof}
Пусть есть два различных предела $a_1, a_2$ последовательности $\{x_n\}$. Найдем из хаусдорфовости отделяющие их окрестности $U_{a_1} \cap U_{a_2} = \emptyset $. 
    
Так как $a_1$ -- предел, то $\exists N_1 \in \mathbb{N} \st \forall n> N_1 \implies x_n \in U_{a_1}$. Аналогично $\exists N_2 \in \mathbb{N} \st \forall n> N_2 \implies x_n \in U_{a_2}$. Но тогда для $n > \max(N_1, N_2) \quad x_n \in U_{a_1} \cap U_{a_2}$, что противоречит предположению о непересечении окрестностей.
\end{Proof}

Цепочка включений аксиом отделимости, очевидно, продолжается.
\begin{Lem}
    $T_1 \subset T_2$.
\end{Lem}

\subsection{Аксиома $T_3$}
\begin{Def}
    [Регулярное пространство $R$]
    Топологическое пространство \topX называется регулярным, если произвольное замкнутое множество можно отделить от не содержащейся в ней точки.
    \[
    \forall A \closed \X \forall x \notin A  \quad \exists U_A, U_x \open \X  \st U_a \cap U_X = \emptyset.
    \] 
\end{Def}

Недочетом регулярности является ее несоответствие логике выполнения включений последующих аксиом в предыдущие. Один из таких случаев показан в следующем примере.
\begin{Ex}
    Тривиальная топология является регулярной, но не Хаусдорфовой.

    Как мы знаем, тривиальная топология вообще не является даже $T_0$, а значит и $T_2$. Однако в ней есть ровно два замкнутых множества: пустое и все множество, которые легко отделяются от несодержащихся в них точках (в одном вообще нет точек, второе просто содержит все точки).
\end{Ex}

\begin{Note}
    Получается, что желаемую нами цепочку $T_0 \subset T_1 \subset T_2$ нельзя продолжить регулярностью. Именно поэтому ее не принято называть $T_3$. 

    Для выполнения соотношения  $T_2 \subset T_3$ добавляют (на самом деле в разной литературе по-разному) дополнительные условия.
\end{Note}

\begin{Def}
    [Аксиома $T_3$]
    Топологическое пространство \topX удовлетворяет третьей аксиоме отделимости (является $T_3$ пространством), если оно является регулярным и также удовлетворяет условию  $T_1$.
\end{Def}

Теперь, когда все точки замкнуты, их легко отделить непересекающимися окрестностями, что дает хаусдорфовость. Более строгое утверждение, продолжающее нашу цепочку, записано в лемме.
\begin{Lem}
\[T_2 \subset T_3\]
\end{Lem}

\subsection{Аксиома $T_4$, нормальность}
\begin{Def}
    [Нормальность]
    Топологическое пространство \topX является нормальным, если можно отделить любые два непересекающихся замкнутых множества.
    \[
        \forall F_1, F_2 \closed \X \; \exists U_{F_1}, U_{F_2} \open \X \st U_{F_1} \cap U_{F_2} = \emptyset.
    \] 
\end{Def}

\begin{Ex}
    Любое метрическое пространство является нормальным.

    Действительно, давайте для произвольных замкнутых множеств $F_1, F_2$ рассмотрим такую функцию $f\st X \to \mathbb{R}$
    \[
        f(x) = \frac{\rho(x, F_1)}{\rho(x, F_1) + \rho(x, F_2)},
    \] где под $\rho(x, A)$ понимается расстояние от точки до множества, которое задается как $\inf\limits_{a\in A} \rho(x,a)$.

    Тогда, в силу непрерывности метрики и арифметических операций, $f$ -- непрерывна. А значит можем рассмотреть в качестве окрестностей, отделяющих $F_1$ от $F_2$ 
    \[
        U_{F_1} = \inv{f}((-\infty, \tfrac12)),\quad U_{F_2} = \inv{f}((\tfrac12, \infty)).
    \] 
    (они открыты, как прообразы открытых; а непересекаемость можно проверить, посмотрев на то, какие значения выдает функция при разных $x$ )
\end{Ex}

И снова оказывается, что нормальность не продолжает цепочку включений $T_{i-1} \subset T_i$. Поэтому, аналогично предыдущему, можно добавить первую аксиому. Оказывается, что этого достаточно.
\begin{Def}
    [Аксиома $T_4$]
    Топологическое пространство \topX удовлетворяет четвертой аксиоме отделимости (является $T_4$ пространством), если оно является нормальным и также удовлетворяет условию  $T_1$.
\end{Def}
    
    Оказывается, что $T_4$ пространства позволяют не только отделить замкнутые друг от друга, но также ввести функциональную отделимость. Это примерно то, что было представлено для доказательства нормальности метризуемых пространств. Только теперь верно для любых $T_4$ пространств.
    \begin{Lem}
        [Урысона]
        В $T_4$ пространствах для любых непересекающихся замкнутых множеств $F_1, F_2$ существует отображение  $f \st X \to I$, что $f(F_1) = \{0\}$ и $f(F_2) = \{1\}$.
        \[
            \forall F_1, F_2 \closed X \st F_1 \cap F_2 = \emptyset \implies \exists f \st X \to I \st f(F_1) = \{0\} \text{ и } f(F_2) = \{1\}.
        \] 
    \end{Lem}
    Доказательство данной леммы оказывается не столь очевидным и требующим неких матанализных выкладок, поэтому не будем его приводить.

    Как уже говорилось, для введенного таким образом $T_4$ пространства включения аксиом продолжаются.
    \begin{Lem}
        \[
        T_3 \subset T_4
        \] 
    \end{Lem}
