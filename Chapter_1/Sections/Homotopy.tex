\section{Гомотопии}
\begin{Intro}
В этом параграфе мы будем пытаться перенести конструкции из топологии в элементы линейной алгебры. Оказывается, в топологическом пространстве есть (в основном) группы, которые к тому же сохраняются при гомеоморфизме. Первыми представителями таких групп являются гомотопические группы. Однако для их построения нам необходимо погрузиться в понятие гомотопии, позволяющее разделять отображения на (в каком-то смысле) одинаковые и различные. Описывая интуитивно, можно говорить о непрерывных деформациях отображений. Прочувствовать эту идею легко на двух путях между двумя точками: они будут одинаковыми, если мы сможем постепенно, сантиметр за сантиметром, сдвигать один путь в направлении другого. Но если между этими путями будет болото, то мы не сможем так сделать.

Вообще, теория гомотопий и гомотопических групп отделяется в целую отдельную науку. Мы лишь коснемся самых основ и плавно подойдем к простейшей из них --- фундаментальной группе, рассматриваемой в следующем параграфе.
\end{Intro}
\subsection{Пространства непрерывных отображений}
Однако для начала нужно понять, над каким пространством мы будем издеваться и кого будем деформировать.

Будем рассматривать теперь всевозможные непрерывные отображения между топологическими пространствами $X$ и $Y$. Такое множество будем называть $\mathcal{C}(X, Y)$. Возникает вопрос: можно ли ввести какую-то топологию на этом множестве?

Конечно, всегда существуют дискретная и антидискретная топологии. Но они не несут большого практического применения.

В качестве первого варианта рассмотрим метрическую топологию, которую можно получить, если $Y$ является метризуемым. 
\begin{Def}
    [Топология равномерной сходимости]
    Топология на $\mathcal{C}(X,Y)$, индуцированная метрикой $\mu(f_1, f_2) = \underset{\tiny x \in X}{\sup} \{\rho(f_1(x), f_2(x))\}$, где $\rho$ --- метрика в топологии $Y$, называется топологией равномерной сходимости.
\end{Def}

Другим, отчасти противоположным, примером топологии на непрерывных отображениях, является топология поточечной сходимости. 
\begin{Def}
    [Топология поточечной сходимости]
    Топология на $\mathcal{C}(X,Y)$, базой которой является множество из всевозможных пересечений множеств  $\{f \in \mathcal{C}(X,Y) \st f(x_i) \in U_i\}$, где $x_1, \dots, x_k \in X$, $U_1, \dots U_k \open Y$.
\end{Def}

Данные названия топологий говорят о свойствах внутри них. Но анализом различных сходимостей все-таки занимается функциональный анализ. Поэтому построим топологию таким образом, чтобы воспользоваться топологическими свойствами.

\begin{Def}
    [Компактно-открытая топология]
    Топология на $\mathcal{C}(X,Y)$, базой которой является множество из всевозможных пересечений множеств  $\{f \in \mathcal{C}(X,Y) \st f(K) \subset U\}$, где $K \subset X$ -- компакт, $U \open Y$.
\end{Def}

\begin{Note}
    В таким образом введенной топологии, в частности, можно брать отрезки в качестве компакта. И получившиеся отображения будут являться путями в пространстве $Y$. Нам практически понадобится именно эта составляющая в ближайших параграфах.
\end{Note}
\subsection{Определение гомотопии и примеры}
Слово гомотопия берет свои корни от греческих слов: homós -- одинаковые, tópos -- место. То есть гомотопия пытается описать одинаковые куски пространства. Вспоминая, что речь в данном параграфе идет про пространство $\mathcal{C}(X, Y)$, можно интуитивно переформулировать это понятие в одинаковость каких-то мест, связанных с образами отображений. Чуть более точно, сформулируем в следующем определении.
\begin{Def}
    [Гомотопия]
    Гомотопией между двумя отображениями $f,g \in \mathcal{C}(X,Y)$ назовем такое непрерывное отображение $H\st X\times I \to Y$, для которого $\forall x\in X \quad H(x, 0) = f(x)$, $H(x,1) = g(x)$. 
\end{Def}

Данное определение можно воспринимать как некоторая деформация между двумя отображениями: в начальный момент времени мы имеем функцию $f$, а в конце --- $g$. Все, что между, нас не очень интересует, кроме того, что там деформация непрерывна.

\begin{Ex}
    Простейшим примером является гомотопия для отображений $f_1, f_2 \in C(\{\mathrm{pt}\}, Y)$. На самом деле каждое из таких отображений лишь выбирает точку в пространстве $Y$, а гомотопия --- это путь между выбранными точками. 
\end{Ex}

 \begin{Note}
     В частности, можно увидеть, что является гомотопией двух путей, закрепленных между двумя точками $x_0$ и $x_1$. 

    По определению, для путей $\alpha, \beta \st I \to Y$ это $H\st I \times I \to Y$, для которого $H(x, 0) = \alpha(x), H(x,1) = \beta(x)$. При этом в силу непрерывности $H(0, t) = x_0, H(1, t) = x_1$, то есть все деформации имеют одно начало и конец.

    Из этого можно легко представить случаи гомотопных путей (Рис. \ref{fig:path_nohomo}) и негомотопным (Рис. \ref{fig:path_homo}). Понятно, что во втором случае непрерывной деформации мешает вырезанный кусок между этими двумя путями. 
\end{Note}

Понятно, что можно говорить вместо $F\st X\times I \to Y$ о параметризованном отображении $F_t \st X \to Y \quad t \in I$. Иногда таким образом оказывается удобнее описывать гомотопии.

\begin{Ex}
    В каком-то смысле \textit{удачным} пространством является выпуклое пространство (то есть то, в котором вместе с двумя точками содержится отрезок между ними). В нем любые два отображения можно соединить, так называемой, линейной гомотопией:
    \[
    H(x,t) = f(x) \cdot t + g(x) \cdot (1-t).
    \] 
    Нетрудно убедиться, что это действительно гомотопия.
\end{Ex}
    
Так как в самом слове гомотопия есть корень homós, означающий одинаковый, то появляется идея, что все гомотопные отображения в каком-то смысле одинаковые. А негомотопные --- отличающиеся. Конечно, как и всегда, строго это формулируется в утверждении про отношение эквивалентности.
 
\begin{Prop}
    Гомотопия является отношением эквивалентности на множестве $\mathcal{C}(X,Y)$. 

    Классы эквивалентности называют гомотопическими классами и обозначают $\pi(X,Y)$.
\end{Prop}
\begin{Task}
    Проверьте выполнение рефлексивности, симметричности и транзитивности.
\end{Task}

\begin{Ex}
    Как мы уже выяснили, в выпуклом пространстве все отображения гомотопны друг другу. А значит там есть ровно один гомотопический класс.
\end{Ex}

Говоря о главной теме всей главы --- гомеоморфизмах, можно снова вспомнить, что вообще-то это очень грубое определение. Для него обратное отображение должно давать вместе с прямым ровно тождественное отображение. Однако если предположить, что для пути с одинаковыми началом и концом (петли) обратным является просто проход в обратную сторону, то даже в таком тривиальном случае композиция с обратным не будет равна тождественному. Однако она точно будет гомотопна ему. Отсюда появляется идея ослабления требований гомеоморфизма.
\begin{Def}
    [Гомотопическая эквивалентность]
    $f\in \mathcal{C}(X,Y)$ называется гомотопической эквивалентностью между $X$ и $Y$, если существует $g\in C(Y,X)$ такое, что  $f\circ g \sim 1_Y$ и $f\circ f \sim 1_X$. 

    Пространства, между которыми можно построить гомотопическую эквивалентность будем называть гомотопически эквивалентными. 
\end{Def}


Гомотопическая эквивалентность чем-то отдаленно напоминает гомеоморфизм: есть непрерывное туда и обратно отображение. Однако в гомеоморфизме обратное отображение --- привычное нам, дающее в композиции с прямым тождественное; а в гомотопическом случае под обратным понимается более общее: такое, что их композиция с прямым лишь гомотопична тождественному. 

Нетрудно понять, что гомотопическая эквивалентность является обобщением гомеоморфизма. Ведь гомотопичность включает в себя равенство. Обратное же, очевидно, неверно, что легко увидеть в случае с путями с началом и концом в одной точке (так называемыми петлями). Складывая петлю с такой же, только пройденной в обратном направлении, мы, конечно, не получим тождественное отображение. Но полученное отображение точно гомотопично тождественному.

Некоторым простейшим частным случаем пространств являются стягиваемые пространства.
\begin{Def}
    [Стягиваемые пространства]
    Пространство $X$ называется стягиваемым, если $\exists x_0 \in X \quad 1_X \sim \mathrm{const}_{x_0}$  (то есть константное отображение совпадает с тождественным).
\end{Def}
На самом деле, по-настоящему, название стягиваемое раскрывается в следующей лемме.
\begin{Lem}
    $X$ --- стягиваемое тогда и только тогда, когда $\exists x_0\in X \st X \sim \{x_0\}$. 
\end{Lem}
\begin{Proof}
    $\implies\quad$ Пространства гомотопически эквивалентны, если есть эквивалентность между ними. В нашем случае подойдет $\mathrm{const}_{x_0}$, так как $\mathrm{const}_{x_0} \circ 1_X \sim 1_X$ и $1_X \circ \mathrm{const}_{x_0} \sim 1_X$ из определения стягиваемого пространства.

    $\Longleftarrow\quad$ Рассмотрим гомотопическую эквивалентность $f\st X \to \{x_0\}$ (которая есть в силу гомотопичности $X\sim \{x_0\}$). Понятно, что это может быть только константное отображение, просто потому что других вариантов отображения в одноточечное пространство нет. То есть  $\exists g \in C(\{x_0\} \to X) \st \mathrm{const}_{x_0} \circ g \sim 1_{\{x_0\}} \text{ и } g \circ \mathrm{const}_{x_0} \sim 1_X$. На самом деле же $g$ просто выбирает точку в $X$. Мы положим $g(x_0) = x_0$. Тогда $g\circ \mathrm{const}_{x_0} = x_0 = \mathrm{const}_{x_0}$ и получаем, что $\mathrm{const}_{x_0} \sim 1_X$.
\end{Proof}

То есть стягиваемость надо понимать именно в смысле этого слова: пространство можно стянуть (сжать, сдавить) в точку. 

\begin{Ex}
    Очевидным примером стягиваемого пространства является любое выпуклое евклидово пространство. В нем есть линейная гомотопия.

    В частности, легко представить себе стягивание шара в его центр. 
\end{Ex}
\subsection{Ретракции}
Вообще, мы уже много произносим слово <<деформация>>, всегда представляя произвольные действия над объектом (пространством). Намного логичнее может казаться оставлять уже стоящую в нужном положении часть на месте. Ну или хотя бы <<гомотопно на месте>>. Эти идеи проявляются в различных понятиях ретрактов.

\begin{Def}[Ретракт, Ретракция]
    Подпространство $A\subset X$ --- ретракт пространства $\X$, если существует такое непрерывное отображение $r\st X \to X$ (называемое ретракцией $\X$), что
    \begin{itemize}
        \item $r(X) = A$, 
        \item $\left. r \right|_A = \mathrm{id}$. 
    \end{itemize}
\end{Def}

\begin{Note}
    Иначе говоря, мы "сжимаем" пространство $\X$ до $A$, требуя, чтобы все точки  $A$ оставались на своих местах.

    Ретракцию можно удобно изобразить на коммутативной диаграмме TODO \ref{diag:retract}. 
\end{Note}

Оказывается, что такое определение хоть и весьма полезно (что мы увидим дальше в одном из предложений), но может быть усилено почти до аналогии на гомотопические эквивалентности.

\begin{Def}
    [Деформационный ретракт]
    Подпространство $A \subset X$ --- деформационный ретракт пространства $\X$, если существует ретракция $r$, гомотопная тождественному отображению на $X$ ($r \sim \mathrm{id}_X$).

    Если к тому же в точках $a\in A$ отображение совпадает с  $\mathrm{id}_X$ (то есть $a$ под действием гомотопии остаются неподвижны), то $A$ называется строгим деформационным ретрактом.
\end{Def}
 
\begin{Ex}
    Понятно, что всякое одноточечное подмножество $\{x_0\} \subset X$ является ретрактом. Достаточно лишь задать отображение $r\st x \mapsto x_0$, которое, как известно, является непрерывным и при этом удовлетворяет условию $r(x_0) = x_0$.

    Однако это отчасти и является очень широким свойством, ведь ничего не говорится о том, каким образом будут происходить эти отображения. 

    Деформационный ретракт же четко задает простоту и адекватность этих деформаций пространства условием гомотопности $\mathrm{id}$.

    К примеру, нетрудно построить сильный деформационный ретракт точки n-мерного диска $r\st \overline{D^n} \to \{\mathrm{pt}\}$. Для этого необходимо предложить гомотопию между $r\st x\mapsto \mathrm{pt}$ и $\mathrm{id}\st \mathrm{pt} \mapsto \mathrm{pt} \in \overline{D^n}$. А это ровно гомотопия, предложенная при исследовании стягиваемости пространств.
\end{Ex}

Следующее предложение предлагает некоторую роль ретракта в науке.
\begin{Prop}
   Пусть подмножество $A\subset X$ является ретрактом. Тогда всякое непрерывное отображение $A\to Y$ в произвольное пространство $Y$ можно продолжить до непрерывного отображения $X\to Y$.
\end{Prop}
\begin{Proof}
    Если есть ретракт $r\st X\to A$, то для произвольного отображения  $f\st A \to Y$ можно предложить отображение $g = f \circ r \st X \to Y$. Оно будет непрерывным как композиция двух непрерывных.
\end{Proof}

А вот деформационный ретракт, как и было обещано, дает сразу гомотопическую эквивалентность.
\begin{Prop}
    Если $A\subset X$ --- деформационный ретракт $X$, то пространства $A $ и $X$ гомотопически эквивалентны.
\end{Prop}
\begin{Proof}
    Фактически в самом определении деформационного ретракта явно указана гомотопия, которая является необходимой для $i_A \circ r \sim \mathrm{id}_X$ (где $i_A$--- вложение, $r$ --- ретракт), а $r \circ i_A = \mathrm{id}_A$.
\end{Proof}

