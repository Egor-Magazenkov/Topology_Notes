\section{Некоторые топологические конструкции}
\begin{Intro}
В данном параграфе мы поговорим про способы создания новых топологических пространств на основе уже имеющихся. Два таких примера уже были разобраны, когда мы говорили про топологию подпространства и топологию произведения. Однако совершенно новую идею дает понятие фактортопологии, которое позволяет описать операции склеивания и стягивания двух пространств. Конструкция, рассматриваемая во втором пункте позволяет из любого пространства сделать компактное добавлением одной точки. А третий пункт приводит интересный пример пространства, которое активно используется физиками и заложено в головы большинству фотографов. В общем, получается очень красивый параграф с огромным количеством примеров, некоторые из которых даже можно будет пощупать в физическом смысле. 
\end{Intro}
\subsection{Конструкции на основе факторпространства}

\begin{Def}
    [Фактортопология]
    Пусть есть топологическое пространство \topX и на множестве $\X$ введено отношение эквивалентности $\sim$. Тогда на множестве (или, как его еще называют, фактормножестве) $\X / \sim$ можно построить топологию:
    \[
        U \open \X / \mathord{\sim} \Leftrightarrow \inv{\pi}(U) \open \X,
    \]где $\pi \st \X \to \X / \sim$
\end{Def}

\begin{Task}
    В том, что определение дано корректно, Вам предлагается убедиться самим.
\end{Task}

\begin{Prop}
    Фактортопология является тончайшей среди всех топологий с непрерывным отображением $\pi$. То есть $\Omega_{\text{остальные}} \subset \Omega_{\X / \sim}$
\end{Prop}

\begin{Lem}
    Фактортопология наследует свойства исходной топологии (компактность, связности, счетности\dots).
\end{Lem}


\begin{Lem}
Пусть дано отображение $f\st \X \to \Y$, что  $f(x) = f(y)$ для  $x\sim y$.    

Тогда существует единственное отображение $\bar{f}\st \X / \sim \to \Y$, причем непрерывность $f$ равносильна непрерывности $\bar{f}$.
\end{Lem}

\begin{Def}
    [Экстраполяция]
    Пусть $f\st \X \twoheadrightarrow \Y$ -- сюръекция, причем  $\X$ -- компакт, а  $\Y$ -- Хаусдорфово.

    Тогда $\bar{f}\st \X / \sim \to \Y$ -- гомеоморфизм. Это свойство называется экстраполяцией. 
\end{Def}

\begin{Ex}
    Рассмотрим отображение $f\st [0,1] \to \mathcal{S}^1$, заданное как $x \overset{f}{\mapsto} (\cos 2\pi x, \sin 2\pi x)$.
    
    Понятно, что оно сюръективно, но не биективно. А значит не может быть и гомеоморфизма.

    Однако, рассматривая фактор $[0,1] / \{0,1\}$ (стягиваем точки 0 и 1 в одну), получим, пользуясь экстраполяцией,
     \[
         [0,1] \overset{\pi}{\to} [0,1] / \{0,1\} \overset{\bar{f}}{\to} \mathcal{S}^1.
    \] 
    И при этом $\bar{f}$ -- гомеоморфизм.
\end{Ex}

\begin{Def}
    [Стягивание]
    Стягиванием $\X / A$ пространства  $\X$ по множеству $A \subset \X$ назовем факторпространство, получаемое из отношения $x\sim y \Leftrightarrow (x=y \text{ или } (x\in A \text{ и } y \in A))$. 
\end{Def}
\begin{Ex}
    Стянем сферу по какой-нибудь из окружностей, секущих ее. Получим две сферы, касающиеся в точке.
\end{Ex}

\begin{Def}
    [Приклеивание]
    Пусть $f\st \X \to  \Y$. Приклеиванием $\X \underset{f}{\sqcup} \Y$ назовем пространство $\X \times [0,1] \sqcup \Y / (x,1) \sim f(x)$.
\end{Def}
\begin{Ex}
    Представим себе сферу, в которой мы спилили кусок напильником, получив дырку. А также тор, в котором таким же образом выпилена другая дырка. А теперь приклеим одну дырку к другой предложенным выше образом. 

    Получим сферу с ручкой, которая, конечно, является тем же тором.

    Однако, приклеивая аналогично к тору другой тор, получим уже сферу с двумя ручками, которая раньше не появлялась. Таким образом можно получать сферы с любым количеством ручек.
\end{Ex}
\begin{Ex}
    Аналогичным образом можно приклеивать ленты Мебиуса к сфере. Получившиеся фигуры называют сферой с пленками.
\end{Ex}

\begin{Def}
    [Конус]
    Конусом $C\X$ над пространством $\X$ назовем $\X \times [0,1] / (x,1) \sim \{\mathrm{pt}\}$.
\end{Def}
\begin{Ex}
    Конус над окружностью --- это привычный нам конус. 
\end{Ex}

\begin{Def}
    [Надстройка]
    Надстройкой $\Sigma \X$ над пространством $\X$ назовем пространство $C\X / \X \times \{0\}$.
\end{Def}

\begin{Ex}
    Надстройка над окружностью дает <<двусторонний конус>>.
\end{Ex}
\subsection{Одноточечная компактификация}
\subsection{Проективное пространство}

