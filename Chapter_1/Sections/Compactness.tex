\section{Компактность}
\begin{Intro}
 Понятие компактности является одним из основных в топологии и используется повсеместно в ее приложениях. На самом деле это некоторое подобие конечности в множествах. Также появляется более мягкое понятие паракомпактности, которое весьма понадобится при изучении многообразий.
\end{Intro}

\subsection{Компактное пространство}
Понятие компакта тесно завязано на покрытиях. Интуитивно понятно, что это такой набор, который \textit{покрывает} полностью исходное множество. Рассмотрим более строгие определения.
\begin{Def}
     [Покрытие]
     Множество $\{S_i\}_{i\in I}$ открытых подмножеств пространства \topX называется покрытием этого пространства, если $\X \subset \underset{\tiny i\in\mathcal{I}}{\cup} S_i$.
\end{Def}
\begin{Def}
    [Подпокрытие]
    Множество $\tilde{S}$ называется подпокрытием для покрытия $S$ пространства \topX, если $\tilde{S}$ -- покрытие для $\X$, и $\tilde{S} \subset S$.
\end{Def}
\begin{Def}
    [Компактное пространство]
    Пространство \topX называется компактным, если из любого его покрытия можно выбрать конечное подпокрытие этого пространства.
\end{Def}
\begin{Note}
    Заметим, что скорее всего Вы уже встречались с понятием компакта в математическом анализе. Там появлялась следующая лемма, показывающая какие множества являются компактными в $\mathbb{R}^n$. Ее мы приведем здесь без доказательства.
\end{Note}
\begin{Lem}
    Пространство $\X \subset \mathbb{R}^n$ является компактным тогда и только тогда, когда  $\X$ замкнуто и ограничено.
\end{Lem}

Иногда вывод о компактности какого-то множества можно сделать не по определению, а пользуясь набором следующих свойств.
\begin{Lem}
    [Свойства компакта]
    Пусть $\X, \Y$ -- компакты, $f\st \X \to \Y$ -- непрерывное.

    Тогда
    \begin{enumerate}
        \item $\X \times \Y$ -- компакт,
        \item Образ функции $f(X)$ -- компакт,
        \item Произвольная функция  $g\st \X \to \mathbb{R}$ достигает своего максимума и минимума,
        \item Замкнутое подмножество $A \closed \X$ является компактом,
        \item Компактное подмножество $A$ Хаусдорфового пространства $\X$ является замкнутым $A \closed \X$.
    \end{enumerate}
\end{Lem}
\begin{Proof}
    \begin{enumerate}
        \item TODO
    \item Рассмотрим некоторое покрытие $f(X) = \cup_{i=1}^\infty U_i$. В силу непрерывности $\cup_{i=1}^\infty \inv{f}(U_i) = X$. При этом $X$ -- компакт, а значит можно выбрать его конечное подпокрытие $\mathcal{U} = \{ U_j \}_{j=1}^{N}$. 

        А $f(\mathcal{U} )$ покрывает образ отображения, что нам и требовалось.
    \item По предыдущему пункту  $f(X)$ -- компакт. Однако единственный компакт в $\mathbb{R}$ -- отрезок. А на нем достигается и минимум и максимум.
    \item TODO
    \item TODO
    \end{enumerate}
\end{Proof}

Очень важной является следующая теорема, позволяющая делать выводы о гомеоморфизме некоторой функции (собственно о том, что мы глобально и хотим исследовать).
\begin{Th}
    Непрерывная биекция $f\st \X \to \Y$ из компакта в Хаусдорфово пространство является гомеоморфизмом.
\end{Th}
\begin{Proof}
    Понятно, что нам достаточно доказать непрерывность обратного $\inv{f}$, то есть 
    \[
        \forall Z \closed X \implies \inv{(\inv{f})} (Z) = f(Z)  \closed Y.
    \] 
    Из 4-ого свойства такое $Z$ является компактом. А тогда из 2-ого свойства и $f(Z)$ --- компакт. И, таким образом, из 5-ого свойства получаем замкнутость $f(Z) \closed Y$.
\end{Proof}

\subsection{Секвенциальный компакт}
В некоторых случаях оказывается удобнее оперировать последовательностями, а не покрытиями. Тогда можно говорить о секвенциальном компакте. Как мы увидим, в привычных нам пространствах это понятие равносильно обычной компактности.
\begin{Def}
    [Секвенциальный компакт]
    Пространство \topX является секвенциально компактным, если из любой последовательности можно выбрать сходящуюся подпоследовательность.
\end{Def}
\begin{Th}
    Если пространство удовлетворяет второй аксиоме счетности, то секвенциальная компактность равносильна компактности.
\end{Th}

\subsection{Паракомпактность}
К сожалению, далеко не все пространства компактны. Но хочется иметь какую-то более слабую альтернативу. Одной из таких важных альтернатив является паракомпактность, с которой связано часто применяющееся понятие разбиения единицы.
\begin{Def}
    [Локальная компактность]
    Пространство \topX называется локально компактным, если около любой точки можно найти компактную окрестность.
\end{Def}
\begin{Def}
    [Локально конечное покрытие]
    Покрытие называется локально конечным, если вокруг любой точки можно найти окрестность, пересекающую конечное число множеств из этого покрытия.
\end{Def}

\begin{Def}
    [Паракомпактное пространство]
    пространство \topX называется паракомпактным, если в любое покрытие можно вписать локально конечное покрытие.
\end{Def}

\begin{Ex}
    Стандартная топология над $\mathbb{R}^n$ является паракомпактной.
\end{Ex}

\begin{Prop}
    Из компактности следует паракомпактность.
\end{Prop}

\section{Разбиение единицы}
Для формулирования понятия разбиения единицы нам понадобится носитель функции --- что-то близкое к множеству, на котором функция не обнуляется.
\begin{Def}
    [Носитель функции]
    Носителем $\mathrm{supp}\,f$ функции $f\st X \to \mathbb{R}$ называется множество $\mathrm{Cl} \{x\in X \st f(x) \neq 0\}$.
\end{Def}

\begin{Def}
    [Разбиение единицы]
    Семейство неотрицательных функций $f_\alpha\st X \to \mathbb{R}_+$ называется разбиением единицы, если
    \begin{enumerate}
        \item $\mathrm{supp}\, f_\alpha$ составляют локально конечное покрытие $X$,
        \item $\sum\limits_{\alpha} f_\alpha(x) = 1$.
    \end{enumerate}
\end{Def}
