\section{Аксиомы счетности}
\begin{Intro}
    В данном параграфе мы рассмотрим следующий набор свойств топологических пространств, которые позволяют их отличать. Это, так называемые, аксиомы счетности. Их основой является желание ответить на вопрос, можем ли мы чем-то ограничить какую-то топологическую структуру сверху. То есть требуется понять, будет ли нечто счетным. А вот что стоит за этим \textit{нечто}, мы и посмотрим. Всего мы рассмотрим три ограничения. Два из них пронумерованы, а вот третье --- как бы немного обособленное и без номера.
\end{Intro}
\subsection{1 аксиома счетности}
\begin{Def}[1 аксиома счетности]
    Говорят, что топологическое пространство \topX удовлетворяет первой аксиоме счетности, если для любой точки $x\in X$ есть не более чем счетная база $\mathcal{B} = \{ V_i \}_{i\in I}$ окрестностей точки  $x$, то есть в каждой окрестности $U_x$ содержится некоторая окрестность из базы. 
\end{Def}

\begin{Prop}
    Любое метризуемое пространство удовлетворяет 1 аксиоме счетности.
\end{Prop}
\begin{Proof}
    Рассмотрим в качестве счетной базы окрестностей точки $x$ семейство шаров с центром в точке $x$ радиусов $\frac{1}{n}$, где $ n\in \mathbb{N}$. 
\end{Proof}

\begin{Ex}
    Топология Зариского не удовлетворяет 1 аксиоме счетности.

    Действительно, если существует база $\mathcal{B} = \{ U_k \}_{k=1}^\infty$ окрестностей точки $x$, то рассмотрим точку $y$ из множества $\mathbb{R} \setminus \left( \{x\} \cup \bigcup\limits_{k=1}^{\infty} (\mathbb{R} \setminus U_k) \right)$. 

    Так как одноточечное множество замкнуто, то $\mathbb{R} \setminus \{y\} \open X$. Но эта окрестность не содержит в себе ни одной из окрестностей $\mathcal{B}$, так как ровно их мы и вычитаем. А значит $\mathcal{B}$ не является базой. То есть пришли к противоречию, а значит никакой счетной базы окрестностей точки быть не может.
\end{Ex}

\subsection{2 аксиома счетности}
\begin{Def}
    [2 аксиома счетности]
    Говорят, что топологическое пространство \topX удовлетворяет второй аксиоме счетности, если у него есть не более чем счетная база.
\end{Def}

\begin{Lem}
    Из 2 аксиомы счетности следует 1 аксиома счетности.
\end{Lem}
\begin{Proof}
    Если есть некоторая счетная база у всего пространства, то можно ее же рассматривать как счетную базу окрестностей точки. 
\end{Proof}

    
\begin{Th}
    [Линделефа]
    Если пространство \topX удовлетворяет 2 аксиоме счетности, то из любого покрытия $X$ можно выбрать не более чем счетное его подпокрытие.
\end{Th}
\begin{Proof}
    Пусть $\mathcal{B} = \{V_k\}_{k\in\mathbb{N}}$ -- счетная база из 2 аксиомы счетности, и $\{U_i\}_{i\in I}$ -- некоторое покрытие $X $.

    Для каждого $V_k$ выберем такое множество $U_{i(k)}$ из покрытия, что $V_k \subset U_{i(k)}$. 

    Тогда $X = \bigcup\limits_{k \in \mathbb{N}} V_k \subset \bigcup\limits_{k \in \mathbb{N}} U_{i(k)}$. При этом также $\bigcup\limits_{k \in \mathbb{N}} U_{i(k)} \subset X $. А значит на самом деле они просто равны, то есть $\{ U_{i(k)} \}_{k\in \mathbb{N}}$ -- необходимое подпокрытие.
\end{Proof}

\begin{Ex}
    Прямая Зоргенфрея не удовлетворяет 2 аксиоме счетности.

    Действительно, открытое множество $[x, x+1)$ для каждой точки должно содержать некоторый элемент базы. Однако для различных точек  $x\neq y$ эти элементы базы отличаются, так как для $x<y \quad x \notin [y, y+1)$ . То есть элементов базы по крайней мере столько же, сколько и точек на прямой, а это точно не счетно.
\end{Ex}

\subsection{Сепарабельность}
\begin{Def}
    [Сепарабельное пространство] 
    Говорят, что топологическое пространство \topX сепарабельно, если у него есть не более чем счетное всюду плотное подмножество.
\end{Def}

\begin{Lem}
    Из 2 аксиомы счетности следует сепарабельность.

    При этом, если пространство метризуемо, то верно и обратное.
\end{Lem}
\begin{Proof}
    Рассмотрим некоторую счетную базу $\mathbb{B} = \{B_i\}_{n=1}^\infty$ и выберем из каждого элемента этой базы по одной точке. Полученное множество назовем $H$ (то есть $H = \{ x_i \st x_i \in B_i\}$). 

    Во-первых, заметим, что $H$ счетно. А во-вторых, оно всюду плотно, так как любая другая окрестность точно содержит в себе элемент базы, а значит и соответствующую  точку $x_i$.

    В случае же метризуемого пространства, для всюду плотного $A\subset X$ рассмотрим множество $\{B_r(x) \st x \in A, \; r \in \mathbb{Q}\}$. Оно счетное, как счетное объединение счетных множеств. Утверждается, что это множество является базой. Для доказательства этого достаточно лишь показать, что для любой точки $x$ открытого множества $U$ (в данном случае достаточно сказать лишь про шары)  есть элемент этого множества, лежащий в $U$ и содержащий $x$.

    Рассмотрим точку $p\in X$ и некоторый шар $B_{\varepsilon}(p)$. В силу плотности $A$ можно найти точку $q\in A$, что $\rho(p,q) < \frac{\varepsilon}{3}$. В силу плотность рациональных чисел между $\frac{\varepsilon}{3}$ и $\frac{2\varepsilon}{3}$ можно найти рациональное число $\delta$. А тогда шар $B_\delta (q)$ содержится в $B_\varepsilon(p)$ и содержит $p$, что и требовалось.
\end{Proof}

Заметим, что данная лемма позволяет нам сказать о неметризуемости прямой Зоргенфрея. Так как в обратном случае сепарабельность с метризуемостью дали бы 2 аксиому счетности. А мы ранее убедились, что она не удовлетворяет 2 аксиоме счетности.
