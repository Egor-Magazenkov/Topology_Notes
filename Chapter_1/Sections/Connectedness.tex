\section{Связность}
\begin{Intro}
    Само слово связность наверняка говорит за себя. Это означает, что пространство не состоит из каких-то кусков. И это также вполне логично вписывается в идеологию инвариантов относительно гомеоморфизмов: множество, состоящее из кусков, точно не одинаковое с цельным множеством (проблема возникает именно в связи с этим разрывом, где портится непрерывность отображения). Однако на самом деле, интуитивное представление о связном пространстве как о том, в котором все точки связаны нитями, на самом деле оказывается даже очень сильным и называется линейной связностью. В этом пункте мы увидим, что можно ввести \textit{более мягкую} связность и она тоже будет инвариантом.
\end{Intro}

\subsection{Связность}
\begin{Def}
    [Связность]
    Топологическое пространство \topX называется связным, если его нельзя представить как объединение двух открытых непересекающихся собственных подмножеств.
    \[
    \forall U, V \open \X \st U\cap V = \emptyset \implies X \neq U \cup V.
    \] 
\end{Def}
Говоря о связности мы всегда будем употреблять слова \textit{разбить множество}, подразумевая именно представление в объединение собственных непересекающихся открытых подмножеств.

\begin{Note}
    Также можно построить определение через замкнутые множества
    \[
    \forall F, H \closed \X \st F\cap H = \emptyset \implies X \neq F \cup H.
    \] 
\end{Note}


Так как мы предполагаем, что связность является инвариантом для гомеоморфизма, то естественным является следующее предложение.
\begin{Prop}
    Образ непрерывного отображения связного множества является связным.
    \[
        f\st \X \to \Y \text{ -- непрерывное, } \X \text{ -- связное} \implies f(\X) \text{ -- связное}.
    \] 
\end{Prop}
\begin{Proof}
    Если образ оказался несвязен $f(X) = U\cup V$, то рассмотрим $\inv{f}(U) \cup \inv{f}(V) = X$, которые разбивают исходное множество на собственные (в силу собственности в образе) открытые (в силу непрерывности) непересекающиеся (в силу $U\cap V = \emptyset$) подмножества. Но этого не может быть в силу связности $X$. Значит и $f(X)$ на самом деле связен.
\end{Proof}

Как обычно, все приличные пространства оказываются связными, что мы не раз наблюдали в обычной геометрии.
\begin{Ex}
    Любое выпуклое подмножество $\mathbb{R}^n$ является связным.
\end{Ex}

Интересное свойство добавляется для связных пространств. Оно упрощает обращение с открытыми и замкнутыми множествами, как бы, разделяя их на несовместимые понятия.
\begin{Lem}
    Топологическое пространство \topX связное тогда и только тогда, когда подмножествами, открытыми и замкнутыми одновременно, являются только $\emptyset$ и $\X$.
\end{Lem}
\begin{Proof}
    $\implies$ Пусть пространство связно и $U \subset X$ -- открыто и замкнуто одновременно (при этом $U \neq X$, $U\neq \emptyset$). 

    Тогда рассмотрим $U \cup (X \setminus U) = X$ --- разбиение множества $X$, причем $X\setminus U \open X$ как дополнение замкнутого, а $U\open X$ по условию. А значит мы представили пространство в виде объединения открытых непересекающихся собственных подмножеств, что противоречит условию связности.
    
    $\Longleftarrow$ Пусть $X$ несвязно, то есть $X = U \cup V$, где $U, V \open X$ --- непересекающиеся собственные подмножества. Однако тогда $V = X \setminus U \closed X$ как дополнение открытого. То есть $V$ открыто и замкнуто, чего не может быть из условия.
\end{Proof}
Заметим, что эта лемма не утверждает, что любое подмножество либо открыто, либо замкнуто. По-прежнему существует подмножества, не являющиеся ни замкнутыми, ни открытыми.


В качестве простейшего примера несвязного пространства можно рассмотреть следующий пример.
\begin{Ex}
    $[1,2] \cup [3,4]$ несвязное.
\end{Ex}
\begin{Note}
    Заметим, что в этом примере множество состояло из двух связных множеств: $[1,2]$ и $[3,4]$ -- но само не являлось связным. Это подводит к идее рассматривать отдельно связные части всего множества. Такие множества будем называть компонентами связности.
\end{Note}
\subsection{Компоненты связности}
\begin{Def}
    [Компоненты связности]
    Компонентой связности назовем наибольшее по включению связное подмножество $ \X$.
\end{Def}

\begin{Note}
    В связи с введенным определением возникает небольшой вопрос об его полезности. Интуитивно хочется, чтобы всегда можно рассматривать пространство как набор компонент связности. А уже с каждой из них работать по отдельности. Этот вопрос закрывает следующее утверждение. Однако для него нам понадобится небольшая очевидная лемма.
\end{Note}
\begin{Lem}
Объединение попарно пересекающихся связных множеств является связным.
\[
    \forall \{U_i \text{ -- связное}\}_{i\in I} \st U_i \cap U_j \neq \emptyset \implies \underset{\tiny i \in I}{\cup} U_i \text{ -- связное}. 
\] 
\end{Lem}
\begin{Proof}
    Если оно не связно, то есть $\underset{\tiny i\in I}{\cup} U_i = V \cup W \st V \cap W = \emptyset$, то каждое из $U_i$ либо лежит внутри $V$, либо внутри $W$. При этом так как $U_i \cap U_j \neq \emptyset$, то на самом деле они все лежат только в одном из $V$ и $W$, а значит другое пустое и никакого разбиения на самом деле нет.
\end{Proof}


\begin{Prop}
    Любое пространство можно разбить на связные компоненты связности, то есть
    \begin{enumerate}
        \item каждая точка содержится ровно в одной компоненте связности,
        \item различные компоненты связности не пересекаются.
    \end{enumerate}
\end{Prop}
\begin{Proof}
    \begin{enumerate}
        \item Рассмотрим произвольную точку $x \in X$ и все связные множества $U_i$, содержащие $x$. Тогда, в силу последней леммы, $\underset{\tiny i\in I}{\cup} U_i$ --- связное и при этом содержит $x$. 

            Таким образом, нашли множество, которое связно и содержит любое другое связное, содержащее $x$. То есть компоненту связности по определению.
        \item Если есть две компоненты связности $A \cap B \neq \emptyset$, то $A\cup B$ тоже связное по лемме. Но тогда нарушается условие максимальности по включению из определения.
    \end{enumerate}
\end{Proof}

\begin{Lem}
    Замыкание связного множества является связным.
    \[
        A \text{ -- связное} \implies \mathrm{Cl}(A) \text{ -- связное}.
    \] 
\end{Lem}
\begin{Proof}
    Пусть замыкание не связно, то есть $\mathrm{Cl}(A) = U\cup V$. Рассмотрим $(A\cap U) \cup (A \cap V) = A \cap (U \cup V) = A \cap \mathrm{A} = A$. Так как $A$ -- связное, то это не может быть разбиением и хотя бы одно из $A \cap U$ и $A \cap  V$ пусто. Но в замыкании $\mathrm{Cl}(A)$ содержатся лишь те точки, которые для любой окрестности имеют пересечение с  $A$, то есть $A \cap U \neq \emptyset$. 
\end{Proof}

\begin{Cor}
    Компоненты связности замкнуты.
\end{Cor}
\begin{Proof}
    Для любой компоненты связности $K$ ее замыкание тоже связно. Но в силу того, что $K$ содержит все связные, содержащие точки $K$, то оно содержит и $\mathrm{Cl}(K)$. То есть оба вложения нашлись и  $K = \mathrm{Cl}((K)$.
\end{Proof}

\subsection{Линейная связность}
Более понятной для нас оказывается линейная связность, то есть возможность соединить две точки пространства некой кривой, которую мы будем называть путем.
\begin{Def}
    [Путь]
    Путем, начинающимся в точке $x$ и заканчивающемся в точке $y$ пространства $\X$, назовем непрерывное отображение $\alpha \st [0,1] \to \X$, что $\alpha(0) = x$ и $\alpha(1) = y$.
\end{Def}
\begin{Def}
    [Линейная связность]
    Пространство \topX называется линейно связным, если между двумя любыми точками существует путь.
    \[
        \forall x, y \in \X \implies \exists \alpha\st [0,1] \to \X \text{ -- непрерывное} \st \alpha(0) = x, \, \alpha(1) = y.
    \] 
\end{Def}
По-прежнему остается инвариантность относительно непрерывного отображения.
\begin{Prop}
    Образ непрерывного отображения линейно связного множества является линейно связным.
\end{Prop}
\begin{Proof}
    Пусть $f\st X\to Y$ --- непрерывное, причем $X$ --- линейно связное. Тогда для любых $x, y \in X$ есть путь $\gamma$. И в качестве пути между $f(x)$ и $f(y) $ рассмотрим отображение $f \circ \gamma$, которое непрерывно как композиция непрерывных и для которого $f \circ \gamma (0) = f(x)$, $f \circ \gamma(1) = f(y)$.
\end{Proof}

Как уже говорилось в вступлении, линейная связность является более сильной относительно связности. А значит стоит доказать следующее утверждение.
\begin{Prop}
    Любое линейно связное пространство является связным.
\end{Prop}
\begin{Proof}
    Пусть  $X$ линейно связно, но не связно. То есть $X = U \cup V$, где $U,V \open X$ --- непересекающиеся собственные подмножества. 

    Рассмотрим произвольные точки $x\in U$, $y \in V$. Тогда в силу линейной связности $\exists \gamma \st [0,1] \to X, \; \gamma(0) = x,\; \gamma(1) = y$ --- непрерывное отображение. 

    $\inv{\gamma}(U) \open [0,1]$ как прообраз открытого. Аналогично $\inv{\gamma}(V) \open [0,1]$. 

    Но тогда $\inv{\gamma}(U) \cup \inv{\gamma}(V) = \inv{\gamma}(\underbrace{U\cup V}_X) = [0,1]$. А значит мы получили разбиение отрезка $[0,1]$ на открытые подмножества, причем, так как  $U\cap V = \emptyset$, то $\inv{\gamma}(U) \cap \inv{\gamma}(V) = \inv{\gamma} (U \cap V) = \inv{\gamma}(\emptyset) = \emptyset$ и, так как $U, V$ --- собственные подмножества, то и их прообразы собственные подмножества уже $[0,1]$. Но известно, что отрезок связен, а значит пришли к противоречию.
\end{Proof}
\begin{Note}
    Заметим, что обратное неверно! В качестве примера рассмотрим, так называемую, топологическую синусоиду.

    \[
        \mathcal{S} = \left\{ \sin \tfrac1x \st x \in (0, 1]  \right\} \cup \left\{ (0,0) \right\}.
    \] 
\end{Note}
\begin{Prop}
    Топологическая синусоида является связной, но не является линейно связной.
\end{Prop}
\begin{Proof}
    Покажем связность $\mathcal{S}$.  $H = \mathcal{S} \setminus \{(0,0)\}$ является линейно связной, а значит и связной. То есть проблема может возникнуть лишь при добавлении нуля. Однако если где $(0,0) \in U \open X$, например, то $U$ также содержит и некоторый кусок $H$. И для этого куска, как уже упоминалось, в силу связности $H$ нельзя найти непустое непересекающееся подмножество (которое уже точно не содержит $(0,0)$), которое добавит $U$ до $H$. А значит и все $\mathcal{S}$ нельзя разбить, то есть оно связно.

    Покажем, что $\mathcal{S}$ не линейно связное от противного. Пусть оно линейно связно, тогда $\forall (x,y) \in \mathcal{S} \quad \exists \gamma \text{ --- путь из $(0,0)$ в $(x,y)$}$.

    Так как $\gamma$ непрерывна, то и $\pi_2 \circ \gamma$ непрерывно (где $\pi_2$ --- проекция на вторую координату), то есть по определению при $\varepsilon = \frac{1}{2}$
    \[
    \exists \delta > 0 \st \forall t < \delta \implies |\pi_2(\gamma(t))| < \frac{1}{2}.   
    \] 
    Если $\forall \tilde\delta \leqslant \delta \implies \gamma(\tilde\delta) = (0,0)$, то рассмотрим аналогично непрерывность в точке $\delta$ и найдем $\delta_1 \st \forall \delta < t < \delta_1 \implies |\pi_2(\gamma(t))| < \frac{1}{2}$. Так будем делать, пока не найдем такую $\delta_i$, что $\exists \tilde\delta < \delta_i \st \gamma(\tilde\delta) \neq (0,0)$. На каком-то этапе она точно найдется, иначе мы получим, что $\gamma$ просто константный нуль, что невозможно, ведь это путь в $(x,y)$. Тогда переобозначим $\delta = \tilde\delta$.

    Рассмотрим точки  $x_n = \frac{1}{\frac{\pi}{2} + 2\pi n}$ и найдем такое $n_0$, что $x_{n_0} < \pi_1(\gamma(\delta)) \neq 0$, где $\pi_1$ --- проекция на первую координату. Это можно сделать в силу ограниченности первой координаты $\mathcal{S}$ и явного решения неравенства $n_0 > \frac{\frac{1}{\pi_1(\gamma(\delta))} - \frac{\pi}{2}}{2\pi}$.

    Так как $\pi_1 \circ \gamma$ тоже непрерывна, то по теореме о промежуточном значении 
     \[
         \text{Для } x_{n_0} \in (\underbrace{\pi_1(\gamma(0))}_0, \pi_1(\gamma(\delta))) \quad \exists \tau \in (0, \delta) \st \pi_1(\tau) = x_{n_0}.
    \] 

    Но $\pi_2(\gamma(\tau)) = \sin \frac{1}{x_{n_0}} = 1$, что противоречит условию $|\pi_2(\gamma(t))| < \frac{1}{2}$. А значит $\mathcal{S}$ --- не линейно связное.
\end{Proof}

    Естественно, по аналогии с обычной связностью для линейно связных пространств тоже можно рассматривать компоненты связности (только теперь уже линейные компоненты связности). 

 И на основе линейной связности и компонент можно сформулировать отношение эквивалентности.
 \begin{Prop}
   Отношение соединения путем двух элементов является отношением эквивалентности.

   В частности, классами эквивалентности являются линейные компоненты связности и они тоже замкнуты.
\end{Prop}

\subsection{Локальная линейная связность}
ПОКА НЕ ЗНАЮ, НАСКОЛЬКО НУЖНО.
