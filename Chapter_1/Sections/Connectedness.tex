\section{Связность}
\begin{Intro}
    
\end{Intro}

\subsection{Связность}
\begin{Def}
    [Связность]
    Топологическое пространство \topX называется связным, если его нельзя представить как объединение двух открытых непересекающихся множеств.
    \[
    \forall U, V \open \X \st U\cap V = \emptyset \implies X \neq U \cup V.
    \] 
\end{Def}

\begin{Note}
    Также можно построить определение через замкнутые множества
    \[
    \forall F, H \closed \X \st F\cap H = \emptyset \implies X \neq F \cup H.
    \] 
\end{Note}

\begin{Prop}
    Образ непрерывного отображения связного множества является связным.
    \[
        f\st \X \to \Y \text{ -- непрерывное, } \X \text{ -- связное} \implies f(\X) \text{ -- связное}.
    \] 
\end{Prop}

\begin{Ex}
    Понятно, что любое выпуклое подмножество $\mathbb{R}^n$ является связным.
\end{Ex}

\begin{Lem}
    Топологическое пространство \topX связное тогда и только тогда, когда подмножествами, открытыми и замкнутыми одновременно, являются только $\emptyset$ и $\X$.
\end{Lem}

\begin{Ex}
    $[1,2] \cup [3,4]$ несвязное.
\end{Ex}
\begin{Note}
    Заметим, что в последнем примере множество состояло из двух связных множеств: $[1,2]$ и $[3,4]$ -- но не являлось связным. Это подводит к идее рассматривать отдельно связные части всего множества. Такие множества будем называть компонентами связности.
\end{Note}
\subsection{Компоненты связности}
\begin{Def}
    [Компоненты связности]
    Компонентой связности назовем наибольшее по включению связное подмножество $ \X$.
\end{Def}

\begin{Note}
    В связи с введенным определением возникает небольшой вопрос об его полезности. Интуитивно хочется, чтобы всегда можно рассматривать пространство как набор компонент связности. А уже с каждой из них работать по отдельности. Этот вопрос закрывает следующее утверждение.
\end{Note}

\begin{Prop}
    Любое пространство можно разбить на связные компоненты связности, то есть
    \begin{enumerate}
        \item каждая точка содержится ровно в одной компоненте связности,
        \item различные компоненты связности не пересекаются.
    \end{enumerate}
\end{Prop}

\begin{Lem}
    Замыкание связного множества является связным.
    \[
        A \text{ -- связное} \implies \mathrm{Cl}(A) \text{ -- связное}.
    \] 
\end{Lem}

\begin{Cor}
    Компоненты связности замкнуты.
\end{Cor}

\subsection{Линейная связность}
\begin{Def}
    [Путь]
    Путем, начинающимся в точке $x$ и заканчивающемся в точке $y$ пространства $\X$, назовем непрерывное отображение $\alpha \st [0,1] \to \X$, что $\alpha(0) = x$ и $\alpha(1) = y$.
\end{Def}
\begin{Def}
    [Линейная связность]
    Пространство \topX называется линейно связным, если между двумя любыми точками существует путь.
    \[
        \forall x, y \in \X \implies \exists \alpha\st [0,1] \to \X \text{ -- непрерывное} \st \alpha(0) = x, \, \alpha(1) = y.
    \] 
\end{Def}

\begin{Prop}
    Любое линейно связное пространство является связным.
\end{Prop}

\begin{Prop}
    Образ непрерывного отображения линейно связного множества является линейно связным.
\end{Prop}

\begin{Note}
    Заметим, что обратное неверно! В качестве примера рассмотрим, так называемую, топологическую синусоиду.

    \[
        \mathcal{S} = \left\{ \sin \tfrac1x \st x \in (0, 1]  \right\} \cup \left\{ (0,0) \right\}.
    \] 
\end{Note}
\begin{Prop}
    Топологическая синусоида является связной, но не является линейно связной.
\end{Prop}

\begin{Note}
    Естественно, по аналогии с обычной связностью для линейно связных пространств тоже можно рассматривать компоненты связности (только теперь уже линейные компоненты связности). 
\end{Note}
 \begin{Prop}
   Отношение соединения путем двух элементов является отношением эквивалентности.

   В частности, классами эквивалентности являются линейные компоненты связности.
\end{Prop}

\subsection{Локальная линейная связность}
ПОКА НЕ ЗНАЮ, НАСКОЛЬКО НУЖНО.
