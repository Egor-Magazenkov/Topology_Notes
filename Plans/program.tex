\documentclass[a4paper,14pt]{extarticle}
\usepackage[utf8]{inputenc} % UTF-8 кодировка
\usepackage[russian]{babel} % Русский язык
\usepackage{indentfirst} % красная строка в первом параграфе в главе
\usepackage{amsmath, amsfonts, amssymb, amsthm} % Набор пакетов для математических текстов
\usepackage{cancel} % зачеркивание для сокращений
\usepackage[pdftex]{graphicx} % вставка рисунков
\usepackage{wrapfig, subcaption} % вставка фигур, обтекая текст
\usepackage{caption} % для настройки подписей
\usepackage{tikz} % рисование
\usepackage{circuitikz}
\usepackage{pgfplots} % графики
\usepackage[unicode, pdftex]{hyperref} % гиперссылки
\usepackage{epigraph}
\setlength{\epigraphwidth}{0.45\textwidth}

\usepackage{import}
\usepackage{pdfpages}
\usepackage{transparent}
\usepackage{xcolor}
\usepackage{multirow}
\usepackage{longtable}

\newcommand{\incfig}[2][1]{%
    \def\svgwidth{#1\columnwidth}
    \import{./figures/}{#2.pdf_tex}
}
\pdfsuppresswarningpagegroup=1


\begin{document}
\centering
\textbf{ \Large Примерный план-программа курса общей топологии (и куска алгебры)}

\textit{\large 04.04.2024 --- 13.04.2024}
\pagenumbering{gobble}

    \begin{longtable}{|p{0.1\linewidth}|p{0.25\linewidth}|p{0.65\linewidth}|}
        \hline
         \bfseries Дата & \bfseries Темы & \bfseries Параграфы\\
        \hline
         \multirow{2}*{04.04} & Основные определения & 
                                         \begin{enumerate} 
                                             \item Топологическое пространство + стандартная топология на $\mathbb{R}$
                                             \item База топологии
                                             \item \textit{Воспоминания:} Метрики и различные шары
                                             \item Порожденная метрикой топология
                                             \item Стандартная топология + сравнение с матанализом
                                             \item Топология на подпространстве, относительность открытости
                                             \item Расположение точек в пространстве
                                         \end{enumerate} \\
        \cline{2-3}
                                  & Непрерывность и гомеоморфизмы & 
                                        \begin{enumerate}
                                             \item Непрерывность: равносильные определения
                                             \item Гомеоморфизм: определение и интуиция
                                             \item Примеры гомеоморфизмов
                                         \end{enumerate} \\
       \hline
        \multirow{4}{*}{06.04} & Топологические конструкции & 
                                        \begin{enumerate}
                                            \item Топология произведения
                                            \item Фактортопология
                                            \item Конструкции на основе фактортопологии
                                        \end{enumerate} \\
       \cline{2-3}
                               & Примеры различных топологий & 
                                        \begin{enumerate}
                                            \item Простейшие примеры
                                            \item Различные склейки: лента Мебиуса, тор, бутылка Клейна\dots
                                            \item $M_g$, $N_g$, ?Проективные пространства?
                                        \end{enumerate}
                                        \\
       \cline{2-3}
                               & Аксиомы отделимости &
                                        \begin{enumerate}
                                            \item $T_0$
                                            \item  $T_1$
                                            \item $T_2$
                                            \item $R\; + \; T_3$ 
                                            \item $T_4$ 
                                        \end{enumerate} 
                                        \\
      \cline{2-3}
                               & Аксиомы счетности & 
                                        \begin{enumerate}
                                            \item Сепарабельность
                                            \item I аксиома счетности
                                            \item II аксиома счетности
                                        \end{enumerate}
                                        \\
      \hline
      \multirow{4}{*}{11.04} & Компактность & 
                                        \begin{enumerate}
                                            \item Компактность
                                            \item Секвенциальный компакт
                                            \item Паракомпактность
                                            \item ?Разбиение единицы?
                                        \end{enumerate} \\
      \cline{2-3}
                             & Связность & 
                                        \begin{enumerate}
                                            \item Связность
                                            \item Линейная связность
                                            \item ?Локальная линейная связность?
                                        \end{enumerate}
                                        \\
      \cline{2-3}
                             & Разговор про петли & 
                                        \begin{enumerate}
                                            \item Введение
                                            \item Ретракты
                                            \item Гомотопии
                                        \end{enumerate}
                                        \\
      \cline{2-3}
                             & \textit{Воспоминания:} группы &
                                        \begin{enumerate}
                                            \item Определение
                                            \item Гомо- и изоморфизмы
                                        \end{enumerate}
                                        \\
      \hline
      \pagebreak
      \hline
      \multirow{2}{*}{13.04} & Фундаментальная группа & 
                                        \begin{enumerate}
                                            \item Определение
                                            \item Односвязность
                                        \end{enumerate}
                                        \\
      \cline{2-3}
                             & Вычисление $\pi_1$ & 
                                        \begin{enumerate}
                                            \item Окружность (без доказательства)
                                            \item $M_g$,  $N_g$
                                            \item ?Теорема ван-Кампена?
                                        \end{enumerate}\\
      \hline

    \end{longtable}
\end{document}
