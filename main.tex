\documentclass[a4paper, 12pt]{book}

\title{Рассуждения об общей и дифференциальной топологии, а также о дифференциальной геометрии}
\author{Магазенков Е. Н.}

\usepackage{itmobook}
\usepackage{import}
\usepackage{xifthen}
\usepackage{pdfpages}
\usepackage{transparent}

\newcommand{\X}{\mathbf{X}}
\newcommand{\topX}{$\left(X,\, \Omega_{X}\right)$}
\newcommand{\Y}{\mathbf{Y}}
\newcommand{\topY}{$\left(Y,\, {\Omega_{Y}}\right)$}
\newcommand{\Z}{\mathbf{Z}}
\newcommand{\topZ}{$\left(Z,\,{\Omega_{Z}}\right)$}

\let\oldforall\forall
\renewcommand{\forall}{\oldforall \, }
\let\oldexist\exists
\renewcommand{\exists}{\oldexist \: }

\newcommand{\st}{\; : \;}

\newcommand{\inv}[1]{#1^{-1}}

\newcommand*{\open}{\raisebox{-0.75ex}{\begin{tikzpicture}
		\draw (0,0) node {$\subset$};
		\draw (0.1em, -0.575ex) circle (0.125em);
\end{tikzpicture}}}
\newcommand*{\closed}{\raisebox{-0.75ex}{\begin{tikzpicture}
			\draw (0,0) node {$\subset$};
			\draw[thick] (0.1em, -0.35ex) -- (0.3em, -0.85ex);
\end{tikzpicture}}}

\newcommand{\cl}[1]{\mathrm{Cl}\left(#1\right)}

\newcommand{\incfig}[2][1]{%
    \def\svgwidth{#1\columnwidth}
    \import{./figures/}{#2.pdf_tex}
}

\newcommand{\image}[2]{
    {\begin{center}
   \incfig[0.8]{#1}
   \captionof{figure}{#2 \label{#1}}  \par
 \end{center}}
}


\newcommand{\mysetminusD}{\hbox{\tikz{\draw[line width=0.6pt,line cap=round] (3pt,0) -- (0,6pt);}}}
\newcommand{\mysetminusT}{\mysetminusD}
\newcommand{\mysetminusS}{\hbox{\tikz{\draw[line width=0.45pt,line cap=round] (2pt,0) -- (0,4pt);}}}
\newcommand{\mysetminusSS}{\hbox{\tikz{\draw[line width=0.4pt,line cap=round] (1.5pt,0) -- (0,3pt);}}}

\renewcommand{\setminus}{\mathbin{\mathchoice{\mysetminusD}{\mysetminusT}{\mysetminusS}{\mysetminusSS}}}

\newcommand{\homeo}{\overset{\text{\tiny homeo}}{\cong}}
\begin{document}
	
	\renewcommand{\contentsname}{\hfillОГЛАВЛЕНИЕ\hfill} 
	\frontmatter
	\titlepage
	
	\doublespacing
	\tableofcontents
	\let\cleardoublepage\clearpage
	\singlespacing
	
	\mainmatter
	
	\pagestyle{style}
	
	
	\chapter{Общая топология}
	\epigraph{Невозможна реальность, которая была бы полностью независима от ума, постигающего её.	\leavevmode
	}{\itshape Анри Пуанкаре\\ французский математик,\\ один из основоположников топологии}
	
    Данная глава посвящена некоторому введению в раздел математики, называющийся топологией. Говоря простыми словами, можно описать вопросы, на которые отвечают топологи, как некий анализ объектов на основе лишь их формы и свойств, без опоры на такие характеристики как длины, углы, площади и т.д. При этом оказывается, что такой взгляд на пространства (будем честны -- именно вокруг понятия топологического пространства будут крутиться дальнейшие рассуждения) появляется повсеместно в кардинально различных сферах математики, в том числе и в курсах математического анализа, с которыми читатель уже наверняка знаком. 

    При этом в названии главы также есть слово \textit{общая}. В понятие общей топологии мы будем включать то, что часто еще называют элементарной топологией; ту часть всей науки, которая практически стала большой частью общематематического языка; ту часть, которая является некоторой базой в изучении всей топологии и построена, в большинстве, на введении понятий и рассмотрении некоторых свойств. Можно сказать, что мы представляем здесь свод некоторых правил (состоящий из определений и связывающих их лемм, теорем, утверждений), регулирующих поведение внутри данных пространств. А такими пространствами является практически любое, известное вам пространство.


	
	\section{Основные определения}
\subsection{Топология и топологическое пространство}
\begin{Def}[Топология]
    Рассмотрим произвольное множество $X$. Множество его подмножеств $\Omega$ называется топологией, если выполнен следующий набор свойств:
    \begin{enumerate}
        \item $\emptyset \in \Omega$, $X \in \Omega$;
        \item $\forall U, V \in \Omega  \implies U \cap V \in \Omega$;
    \item $\forall \alpha \in \mathcal{A} \; U_\alpha \in \Omega \implies \underset{\tiny \alpha\in\mathcal{A}}{\cup}U_\alpha \in \Omega $.
    \end{enumerate}
\end{Def}

Расшифровывая данное определение, можно просто запомнить, что пустое и всё множество лежат в топологии (1-ое свойство), конечное пересечение множеств топологии лежит в топологии (2-ое свойство) и любое объединение множеств топологии лежит в топологии (3-ье свойство).
\begin{Note}
    Такое определение появилось неслучайно. Дело в том, что топология как наука создавалась достаточно поздно (в истории развития математики). Поэтому в других частях математики (особенно в матанализе) уже были построены некоторые идеи, которые при развитии топологии хотелось оставить действующими и в ней ради целостности математики.

    Скорее всего, именно поэтому лишь конечное пересечение лежит в топологии. Ведь в матанализе нетрудно найти пример, в котором достаточно хороший набор подможеств -- интервалы -- не содержит в себе какое-то бесконечное пересечение. 
\end{Note}
\begin{Task}
    Попробуйте самостоятельно подобрать такой бесконечный набор интервалов, пересечение которого не будет являться интервалом.
\end{Task}

Понятно, что топология $\Omega$ не существует отдельно от множества $\X$. Именно поэтому правильнее будет рассматривать именно пару множество--топология, которая образует пространство. 

\begin{Def}
    [Топологическое пространство]
    Пара \topX множества $X$ с введенной топологией $\Omega_X$ называется топологическое пространство.
\end{Def}
\begin{Note}
    Так как большая часть последующих размышлений посвящена топологическим пространствам, то часто в дальнейшем мы будем опускать пару множество-топология и ограничимся лишь чуть более жирным написаниемисходного множества -- $\X$.

    То есть под $\X$ стоит понимать как само множества, так и топологическое пространство \topX.
\end{Note}

\begin{Ex}
    Рассмотрим пример, так называемой, стандартной топологии, постоянно использующейся в одномерном  математическом анализе.

    Пусть в качестве множества  $X$ будет числовая прямая $\mathbb{R}$, а в качестве топологии $\Omega$ будет пустое множество и всевозможные объединения интервалов (в том числе и с бесконечными концами). Коротко это можно записать $\left( \mathbb{R}, \, \left\{ \cup(a,b) \st a,b \in \overline{\mathbb{R}} \right\} \right)$.
\end{Ex}
\begin{Task}
    Проверьте свойства топологии из определения для стандартной топологии.
\end{Task}

На самом деле стандартную топологию можно также задавать для высших размерностей и в общем случае для $\mathbb{R}^n$. Об этом поговорим позже в пункте про индуцированные метрикой топологии.

Следующие определения, на самом деле, просто вводят новые названия уже существующим объектам. Это необходимо из-за постоянного обращения к этим объектам в будущем.

\begin{Def}
    [Открытые и замкнутые множества]
    Пусть дано некоторое топологическое пространство \topX.
    \begin{enumerate}
        \item Элементы топологии $\Omega_X$  будем называть открытыми множествами,
        \item Множества $X\setminus U$, где $U\in \Omega_X$, будем называть замкнутыми множествами.
    \end{enumerate}
\end{Def}
\begin{Note}
   Заметьте, что открытые и замкнутые множества (как и в русском языке слова открытое и замкнутое) вовсе не являются противоположными.

   Так, множество может быть 
   \begin{itemize}
       \item \textit{и открытым, и замкнутым}, как пустое и само $X$ (но не всегда только они),
       \item \textit{открытым, но не замкнутым}, как интервал в стандартной топологии,
       \item \textit{замкнутым, но не открытым}, как отрезок в стандартной топологии,
       \item \textit{ни замкнутым, ни открытым}, как полуинтервал в стандартной топологии.
   \end{itemize}
\end{Note}

\begin{Note}
    На самом деле открытые и замкнутые множества являются даже весьма схожими объектами. Так, топологию можно определять через замкнутые множества. Для этого нужно немного модернизировать определение:

    Рассмотрим совокупность $\tilde{\Omega}$ подмножеств множества $X$, для которой 
    \begin{enumerate}
        \item $\emptyset \in \tilde\Omega$, $X \in \tilde\Omega$ --- эти условия никак не меняются (ведь мы знаем, что эти множества одновременно открыты и замкнуты),
        \item $\forall F, G \in \tilde\Omega \implies F \cup G \in \tilde\Omega$,
        \item $\forall \alpha \in \mathcal{A} \; F_\alpha \in \tilde\Omega \implies \underset{\tiny \alpha\in\mathcal{A}}{\cup}F_\alpha \in \tilde\Omega $.
    \end{enumerate}
    Тогда $\tilde\Omega$ описывает всевозможные замкнутые множества $X$, а топологией можно назвать $\Omega = \left\{ A\subset X \st X \setminus A \in \tilde\Omega \right\}$.
\end{Note}

\subsection{База топологии}
Мы примерно разобрались с тем, что такое топология. Однако у нас до сих пор нет никакого способа описания топологического пространства, не описывая всевозможные открытые множества. По этой причине предлагается следующий объект, позволяющий описать некоторую часть всей топологии, которой будет достаточно для восстановления всей структуры.
\begin{Def}
    [База топологии]
    Назовем совокупность $\mathbb{B}$ открытых множеств $\X$ базой топологии \topX, если всякое непустое открытое множество этой топологии можно представить в виде объединения элементов этой совокупности.
\end{Def}
\begin{Ex}
    Так в качестве базы стандартной топологии можно рассмотреть множество всевозможных интервалов с вещественными концами. 
    \[
        \mathbb{B}_{\text{ст}} = \left\{ (a, b) \st a,b \in \mathbb{R} \right\}.
    \] 
    Заметьте, что в отличие от того, как мы вводили эту топологию раньше, мы теперь не рассматриваем лучи $(-\infty, b)$ и $(a, +\infty)$. Однако понятно, что их нетрудно получить объединением, используя интервалы.

    Аналогично, можно рассмотреть только интервалы с рациональными концами. Такое множество тоже будет базой.
\end{Ex}

\begin{Task}
    Подумайте, могут ли различные топологические структуры иметь одну и ту же базу?
\end{Task}

Думаю, что также можно заметить, что какие-то базы могут порождать одни и те же топологии. Для различия баз определим, когда можно говорить про базы, как про одинаковые объекты.
\begin{Def}
    [Эквивалентные базы]
    Базы называются эквивалентными, если они порождают одну и ту же топологию.
\end{Def}

   \subsection{Метрика и ее связь с топологией}
   Наверняка вы уже встречались ранее с понятием метрики на различных курсах по математике (а может и не только). Однако для строгости изложения и в целях напоминания приведем некоторые отрывки из теории метрических пространств.
   \begin{Def}
       [Метрика]
       Функция $\rho\st X \times X \to \mathbb{R}$ называется метрикой, если выполнено
       \begin{enumerate}
           \item $\forall x, y \; \rho(x,y) \geqslant 0$, причем $\rho(x,y) = 0 \Leftrightarrow x = y$,
           \item $\forall x,y \; \rho(x,y) = \rho(y,x)$,
           \item $\forall x,y,z \; \rho(x,y) + \rho(y,z) \geqslant \rho(x,z)$.
       \end{enumerate}
   \end{Def}
   \begin{Def}
       [Метрическое пространство]
       Множество $X$ с введенной на нем метрикой $\rho$ образует метрическое пространство $(X, \rho)$.
   \end{Def}

   \begin{Def}
       [Шары и сферы]
       В метрическом пространстве $(X, \rho)$ для точки $a\in X$ и произвольного положительного вещественного числа $r\in \mathbb{R}_+$ вводятся понятия:
       \begin{enumerate}
           \item Открытого шара $B_r(a)$
               \[
                   B_r(a) = \left\{ x\in X \st \rho(a, x) < r \right\},
               \] 
           \item Замкнутого шара $\overline{B}_r(a)$ 
               \[
                   \overline{B}_r(a) = \left\{ x\in X \st \rho(a, x) \leqslant r \right \},
               \] 
           \item Сферы $S_r(a)$ 
               \[
                   S_r(a) = \left \{ x \in X \st \rho (a, x) = r \right \}.
               \] 
       \end{enumerate}
   \end{Def}

   \begin{Note}
       Важно понимать, что термины <<шар>>, <<сфера>> не всегда передают реальную форму шаров и сфер. 

       Так, как бы странно это не звучало, при разных введенных метриках в $\mathbb{R}^2$ шар может оказаться квадратом, ромбом (при этом в самом привычном нам случае он в действительности окажется шаром, только двумерным, то есть кругом).
   \end{Note}
   \begin{Task}
       Найдите примеры метрик в $\mathbb{R}^2$ с разными формами шаров. 
   \end{Task}

   Оказывается, что в метрическом пространстве всегда есть одна понятная топология, которую мы будем называть метрической топологией.

   \begin{Def}
       [Метрическая топология]
       Множество всевозможных шаров некоторого метрического пространства является базой некоторой топологии. Такая топология называется порожденной метрикой топологией или просто метрической топологией.
   \end{Def}
   \begin{Ex}
       Простейшим примером такой топологии является стандартная топология. Только теперь мы можем описать ее не только для одномерного случая.

       Стандартной топологией для $\mathbb{R}^n$ будем называть топологию, индуцированную евклидовой метрикой.
   \end{Ex}

    В метрической топологии можно немного по-другому рассматривать открытость множества. Именно таким образом обычно обходят страшную \textit{топологию} в курсах математического анализа.

   \begin{Prop}
       В порожденной метрикой топологии множество является открытым тогда и только тогда, когда оно содержит каждую свою точку вместе с некоторым шаром, центром которого она является.
   \end{Prop}
   \begin{Proof}
       $\implies$ Пусть множество $A$ открыто. Тогда оно является объединением некоторых шаров  $A = \underset{\tiny \alpha\in\mathcal{A}}{\cup}B_{r_\alpha}(y_\alpha)$.   

       Для произвольной точки $x \in A$ найдем тот из этих шаров, которому она принадлежит. Пусть это просто $B_r(y)$. Тогда шар $B_{r-\rho(x,y)} (x)$, где $\rho$ -- метрика, является искомым.

       $\Longleftarrow$ Рассмотрим для каждой точки $x$ шар $B_r(x)$ из условия. Тогда $\underset{\tiny x\in A}{\cup}B_r(x) = A$ и при этом, так как все шары открыты, то это объединение открытых множеств --- а значит открытое.
   \end{Proof}

   \begin{Task}
       Проверьте, что замкнутые шары являются замкнутыми множествами, а открытые шары -- открытыми множествами.
   \end{Task}

   Некоторые топологические пространства могут быть порождены метрикой, даже если мы этого не подозреваем (или просто определяем без отсылок к ней). Однако такие топологии образуют группу, которая имеет свои преимущества и упрощения перед остальными топологиями.

   \begin{Def}
       [Метризуемые пространства]
       Топологическое пространство называется метризуемым, если его топологическая структура порождается некоторой метрикой.
   \end{Def}

   \begin{Note}
       Отметим, что далеко не все топологии являются метризуемыми. Простейшим (но не самым показательным) примером неметризуемого пространства является антидискретная топология, состоящая из более чем одной точки.
   \end{Note}
   \subsection{Топология на подпространстве}
   Можно пробовать строить топологию на основе уже имеющихся. Простейшие варианты мы рассмотрим в ближайших двух пунктах, а более конструктивные способы будут представлены в 7 и 8 параграфах. TODO links
   \begin{Def}
       [Индуцированная топология]
   Рассмотрим некоторое подмножество $A\subset X$ пространства \topX. Совокупность  $\Omega_A = \left\{ A \cap U \st U \in \Omega_X \right\}$ является топологией в множестве $A$. Такую топологию называют индуцированной в $A$ топологией.
   \end{Def}
    \begin{Task}
        Проверьте, что индуцированная топология действительно является топологией по определению.
    \end{Task}

    \begin{Prop}
        Множество $F$ является замкнутым в подпространстве $A\subset X$ тогда и только тогда, когда $F = A \cap E$, где $E$ -- замкнуто в $X$.
    \end{Prop}
    \begin{Proof}
        $\implies$ Пусть $F \closed A$. Тогда $A\setminus F \open A$ и это  множество представимо в виде $A \setminus F = A \cap U = A \cap (X \setminus E) = A \cap X \setminus A \cap E = A \setminus A\cap E$, где $U \open X$, $E \closed X$. Избавляясь с двух сторон от $A$, получаем искомое.

        $\Longleftarrow$ Пусть $F = A\cap E$. Положим $U = X \setminus E \open X$. Тогда $A \cap U = A \cap (X\setminus E) = A\cap X \setminus A \cap E = A \setminus F$. Но $A \cap U \open A$, а значит и $A \setminus F$. А тогда $F \closed A$.
    \end{Proof}
    
    \begin{Note}
        Заметим, что множества, являющиеся открытыми в подпространстве вовсе не всегда открыты в объемлющем пространстве.

        Так, рассмотрим стандартную топологию на $\mathbb{R}$ как индуцированную из топологии на $\mathbb{R}^2$. Единственным открытым множеством из $\mathbb{R}$, которое открыто в $\mathbb{R}^2$ будет пустое множество. 

        Такое свойство часто называют относительностью открытости.
    \end{Note}

 Однако иногда все же открытость в подпространстве равносильна открытости в объемлющем пространстве. Рассмотрим это в следующем предложении.

    \begin{Prop}
        Открытые множества открытого подпространства являются открытыми и во всем пространстве.
        \[
        A \open \X  \implies \forall U \open A \quad U \open \X.
        \] 
        или еще проще
        \[
        A \open \X \implies \Omega_A \subset \Omega_X.
        \] 
    \end{Prop}
    \begin{Proof}
        Пусть $U\open A$. Тогда по определению $U = A \cap V$, где $V \open X$. И получается, что, так как $A \open X$, то $U$ есть объединение двух открытых в $X$. А значит оно само открыто и $U \in \Omega_X$. 
    \end{Proof}

    \subsection{Топология произведения}
    Вспомните идею при построении декартова произведения множеств. Фактически мы предъявляем упорядоченную пару. Аналогично можно построить топологическое пространство по двум (или нескольким) топологиям. 
    \begin{Def}
        [Топология произведения]
        Рассмотрим два топологических пространства \topX и \topY. Тогда на декартовом произведении  $X \times Y$ можно рассмотреть топологию, порожденную базой
         \[
             \mathbb{B} = \{ U \times V \st U \open X, \; V \open Y\}
        \] 
    \end{Def}
    \begin{Ex}
        Заметим, что стандартная топология на $\mathbb{R}^2$, к примеру, совпадает с топологией произведения стандартных топологий на $\mathbb{R}$. 
    \end{Ex}

    \subsection{Расположение точек относительно множества}
    Пока у нас нет никакого способа определять открытые множества без разложения в объединение других открытых. Как один из таких вариантов, можно рассматривать различные точки и окрестности вокруг них. На основе этих окрестностей можно определить разные классы точек пространства. 
    \begin{Def}
        Пусть \topX --- топологическое пространство,  $A \subset X$. Точка $b\in A$ называется:
        \begin{enumerate}
            \item внутренней для множества $A$, если есть окрестность этой точки, полностью лежащая в  $A$
                \[
                    b \text{ -- внутренняя, если } \exists U \ni b \st U \subset A.
                \] 
            \item внешней для множества $A$, если есть окрестность этой точки, не пересекающаяся с $A$
                \[
                    b \text{ -- внешняя, если } \exists U \ni b \st U \cap A  = \emptyset.
                \] 
            \item граничной для множества $A$, если любая окрестность этой точки, пересекается с $A$ и с $X \setminus A$
                \[
                    b \text{ -- граничная, если } \forall U \ni b \implies U \cap A  \neq \emptyset \text{ и } U \cap (X \setminus A) \neq \emptyset.
                \] 
        \end{enumerate}
    \end{Def}
    Понятно, что все внутренние точки образуют некоторое множество. Интересно, что это множество можно задавать и другим способом. 
    \begin{Def}
        Пусть \topX --- топологическое пространство,  $A \subset X$. Внутренностью $\mathrm{Int}(A)$ множества $A$ называется:
          \begin{enumerate}
              \item множество его внутренних точек,
              \item объединение всех открытых множеств, лежащих в $A$. 
          \end{enumerate} 
    \end{Def}
    \begin{Prop}
        Определения внутренности эквивалентны.
    \end{Prop}
    \begin{Proof}
    \end{Proof}

    Так как мы увидели, что внутренность является открытым множеством, то появляется способ определения открытых множеств.
    \begin{Prop}
        $A \open X$ тогда и только тогда, когда $\mathrm{Int} (A) = A$.
    \end{Prop}
    \begin{Proof}
        $\implies$ Так как множество открыто, то $\mathrm{Int} (A) = \underset{\tiny U\open X, U\subset A}{U} = A$, так как $A$ само одно из этих множеств в объединении.

        $\Longleftarrow$ Очевидно, так как $\mathrm{Int} (A)$ открыто как объединение открытых.
    \end{Proof}

    Аналогично можно ввести понятие внешности. 

    Говоря про классификацию точек, можно рассмотреть немного другой подход. 
    \begin{Def}
        Пусть \topX --- топологическое пространство,  $A \subset X$. Точка $b\in A$ называется:
        \begin{enumerate}
            \item точкой прикосновения для $A$, если любая окрестность пересекается с $A$
                \[
                \forall u \ni b \implies u \cap A \neq \emptyset.
                \] 
            \item предельной точкой, если любая проколотая окрестность пересекается с $A$
                \[
                    \forall u \ni b \implies u \cap (A \setminus \{a\}) \neq \emptyset.
                \] 
        \end{enumerate}
    \end{Def}
    С такими точками тоже можно ввести некие множества.
    \begin{Def}
        Замыканием $\mathrm{Cl(A)}$ множества $A \subset X$ называется множество его точек прикосновения.
    \end{Def}
    \begin{Prop}
        Замыкание равно пересечению всех замкнутых множеств, содержащих $A$.
    \end{Prop}
    \begin{Task}
        Докажите данное утверждение аналогично утверждению про внутренность.
    \end{Task}

    \begin{Prop}
        $A \closed X$ тогда и только тогда, когда $\mathrm{Cl}(A) = A$.
    \end{Prop}
    \begin{Task}
        Докажите данное утверждение аналогично утверждению про внутренность.
    \end{Task}

    Мы знаем из курса матанализа, что множество рациональных чисел всюду плотно в множестве вещественных. Там это выражалось в смысле: между любыми двумя вещественными числами можно найти рациональное. Однако плотность можно вводить, используя замыкание, для любых пространств.
    \begin{Def}
        Пусть $A, B \subset X$. Говорят, что
        \begin{enumerate}
            \item $A$ плотно в $B$, если $B \subset \mathrm{Cl}(A)$
            \item $A$ всюду плотно в $X$, если $\mathrm{Cl}(A) = X$.
        \end{enumerate}
    \end{Def}
    \begin{Ex}
        Как уже говорилось, $\mathbb{Q}$ всюду плотно в $\mathbb{R}$.

        Также $\mathbb{I}$ всюду плотно в $\mathbb{R}$. 
    \end{Ex}

    \subsection{Последовательности}
    Как и в матанализе, можно рассматривать последовательности из точек пространства. Определение последовательности и предела совершенно не отличается от привычного. Однако оказывается, что, в отличие от стандартной топологии, не всегда предел единственный.
    \begin{Def}
        [Последовательность]
        Последовательностью в пространстве $X$ назовем отображение $q\st \mathbb{N} \to X$.
    \end{Def}
    \begin{Def}
        [Сходящаяся последовательность]
        Говорят, что последовательность $\{x_n\}_{n=1}^\infty$ сходится к $a \in X$, если \[
            \forall U(a) \open X \implies \exists N \in \mathbb{N} \st \forall n > N \implies x_n \in U(a).
        \] 
    \end{Def}
    \begin{Ex}
        Рассмотрим топологию Зариского и последовательность $\{n\}_{n=1}^\infty$. Любое натуральное число является пределом такой последовательности.
    \end{Ex}
    Данный пример показывает, что не всегда предел единственный. Мы еще увидим далее, какого свойства будет достаточно для единственности предела. 

	\section{Непрерывные отображения}
\subsection{Непрерывность}
\begin{Note}
    Ниже приведены 4 определения непрерывности отображения. На самом деле наплодить определений можно еще много, тут приведены наиболее распространенные. Более того, отдельной задачей будет показать, что все определения эквиваленты, то есть определяют одно и то же понятие.
\end{Note}
\begin{Def}
    Пусть даны два топологических пространства \topX и \topY, а также теоретико-множественное отображение $f: \X \to \Y$. 
    \begin{enumerate}
        \item 
                $f$ называется непрерывным в точке $x\in \X$, если 
            \[
                \forall U({f(x)})\open \Y \implies \exists V(x) \open \X \st f(V(x)) \subset U(f(x)).
            \] 
                Будем говорить, что $f$ непрерывное, если оно непрерывно в каждой точке~$\X$.
        \item Будем говорить, что $f$ непрерывное, если прообраз любого открытого открыт, то есть
            \[
                \forall U \open \Y \implies \inv f (U) \open \X.
            \] 
        \item Будем говорить, что $f$ непрерывное, если прообраз любого замкнутого замкнут, то есть
            \[
                \forall F \closed \Y \implies \inv f (U) \closed \X.
            \] 
        \item Будем говорить, что $f$ непрерывное, если образ замыкания лежит в замыкании образа, то есть
            \[
                \forall A \subset \X \implies f(\cl A) \subset \cl{f(A)}.
            \] 
    \end{enumerate}
\end{Def}

Соответственно, ниже приведено обещанное утверждение, показывающее, что данные определения определяют одно и то же.

\begin{Lem}
    Определения непрерывности 1-4 эквивалентны.
\end{Lem}
\begin{Proof}
    Докажем в порядке $1 \implies 4 \implies 3 \implies 2 \implies 1$.

    \noindent $1 \Rightarrow 4$ Пояснительный рисунок к доказательству смотри на Рис. \ref{fig:lem1_1to4}.\\
    Рассмотрим произвольную точку $x\in \cl A$ и произвольную $U(f(x))$.\\
    Тогда из 1: $\exists V(x) \st f(V(x)) \subset U(f(x))$.\\
    Так как $x$ лежит в замыкании $A$, то $\exists a\in A \cap V(x)$.\\
    А значит $f(a) \in f(A) \cap f(V(x)) \subset f(A) \cap U(f(x))$.\\
    Учётом произвольности выбора окрестности $U(f(x))$  для любой точки $f(x) \in f(\cl A)$ верно, что $f(x) \in \cl f(A)$.

    {\begin{center}
        \incfig[0.8]{lem1_1to4}
        \captionof{figure}{Пояснительная картинка к переходу $1\implies 4$ \label{fig:lem1_1to4}}
    \par
    \end{center}}

    \noindent $4 \Rightarrow 3$\\
    Пусть $F\closed Y$, но $\inv{f}(F)$ не замкнуто.\\
    Рассмотрим точку $d\in \cl{\inv f(F)} \setminus \inv f(F)$.\\
    Тогда $f(d) \in f(\cl{\inv f(F)}) \underset{\text{по 4}}{\subset} \cl{f(\inv f(F))} = \cl{F} = F$.\\
    Однако получается, что  $d\in \inv f (f(d))) \subset \inv f(F)$, что противоречит предположению $d\in \cl{\inv f(F)}\setminus \inv f(F)$.\\
    А значит исходное предположение было неверно и $\inv f (F) \closed X$.

    \noindent$3\Rightarrow 2$ \\
    Пусть $F\closed Y$. Тогда по 3: $\inv f(F) \closed X$.\\
    Рассмотрим $U=Y\setminus F \open Y$.\\
    При этом $\inv f(U) = \inv f(Y\setminus F) = \inv f(Y) \setminus \inv f(F)  = X \setminus \inv f(F)\open X$.\\
    То есть $\inv f(U) \open X$.

    \noindent$2\Rightarrow 1$ \\
    Рассмотрим произвольную точку $x\in X$ и соответствующую ей окрестность $U(f(x)) \open Y$.\\
    По 2 $\inv f(U(f(x))) \open X$.\\
    Так как $x\in \inv f(U(f(x))) $, то можем рассмотреть $V(x) = \inv f(U(f(x)))$.
\end{Proof}

\begin{Lem}
    Пусть $f\st \X \to \Y$ и $g\st \Y \to \Z$ --- непрерывные отображения.

    Тогда отображение $g\circ f \st \X \to \Z$ также является непрерывным.
\end{Lem}
\begin{Proof}
    Рассмотрим $U\open \Z$.\\
    Так как $g$ -- непрерывное отображение, то $V = \inv g(U) \open \Y$.\\
    Так как $f$ -- непрерывное отображение, то $O = \inv f (V) \open \X$.\\
    При этом получаем, что $O = \inv f(\inv g(U)) = \inv f \circ \inv g = \inv{(g\circ f)} \open \X$.\\
    А значит по 2 определению $g\circ f$ непрерывное.
\end{Proof}

\begin{Ex}
    Отображение $\mathrm{id} \st X \to X$, что $\forall x\quad \mathrm{id}(x) = x$, является непрерывным. 
\end{Ex}
\begin{Ex}
    Константное отображение $\mathrm{const}_c \st X \to Y$ $\mathrm{const}_c(x) = c$ является непрерывным.
\end{Ex}
\begin{Task}
    Рассмотрим отображение $f \st [0,2] \to [0,2]$, $f(x) = \begin{cases}
        x, \quad x\in [0, 1)\\ 3-x, \quad x\in [1,2]
    \end{cases}$. Найдите открытое множество, прообраз которого не является открытым. 

    Таким образом, данное отображение не является непрерывным.
\end{Task}

Непрерывность -- это хорошее свойство, однако оказывается, что, чтобы сравнивать между собой пространства, необходимо более сильное свойство, на которое мы и посмотрим в следующем пункте.
\subsection{Гомеоморфизм}
\begin{Def}
    Пусть даны два топологических пространства \topX и \topY. 
    \begin{enumerate}
        \item Отображение $f: \X\to \Y$ называется гомеоморфизмом (\textit{homeomorphism}), если оно биективное, а также $f$ и $\inv f$ -- непрерывные.
        \item Если между пространствами $\X$ и $\Y$ можно построить гомеоморфизм, то такие пространства называют гомеоморфными. 
    \end{enumerate}
\end{Def}
\begin{Note}
    Будем обозначать гомеоморфные пространства символом $\homeo$.
\end{Note}
\begin{Note}
    Интуитивно, можно воспринимать гомеоморфность двух пространств как возможность деформировать сжатием или растяжением одно пространство в другое (ну и, соответственно, обратно). Важно, что эта деформация происходит без разрезов и склеиваний. 
\end{Note}

\begin{Lem}
    Отношение $\homeo$ является отношением эквивалентности.
\end{Lem}
\begin{Proof}
    Рассмотрим произвольные топологические пространства \topX, \topY и \topZ.
    \begin{itemize}
        \item Рефлексивность: $\X\homeo \X$ -- следует из того, что тождественное отображение непрерывно.
        \item Симметричность: $\X \homeo \Y \implies \Y \homeo \X$ -- следует из того, что гомеоморфизм биективен. 
        \item Транзитивность: $\X \homeo \Y, \Y \homeo \Z \implies \X \homeo \Z$ -- следует из непрерывности композиции. 
    \end{itemize}
\end{Proof}

Таким образом, топологические классы разбиваются на классы эквивалентности. Во многом, именно интерес в определении гомеоморфности двух пространств и развивал науку топологию. 

А так как определение негомеоморфности двух пространств требует доказательства несуществования гомеоморфизма, что является задачей, которую непонятно как решать, то начали рассматривать некоторые свойства, которые сохраняются при пропускании через любой гомеоморфизм. Это давало возможность находить различия в этих инвариантах и утверждать о негомеоморфности пространств.

Однако интереснейшим вопросом оказалась задача нахождения набора свойств, которым должны удовлетворять два множества, чтобы можно было утверждать, что пространства являются гомеоморфными. И, к сожалению (или, может, к счастью), оказалось, что такого набора инвариантов не существует.

На самые используемые инварианты мы посмотрим в следующих главах, а пока давайте рассмотрим несколько примеров гомеоморфизмов.

\subsection{Примеры гомеоморфизмов}

\

\begin{Ex}
    $[0,1] \homeo [a,b]$.

    Построим гомеоморфизм, который легко показать на рисунке (см. Рис. \ref{homeo_linear}) и не менее просто записать явно.
    \image{homeo_linear}{Пояснительная картинка к построению гомеоморфизма между отрезками}
    Так, прямое отображение 
    \[
        \begin{split}
            f: [0,1] &\to [a,b]\\
            x &\mapsto (b-a)x + a.
    \end{split}
    \] 

    И обратное 
    \[
    \begin{split}
        g: [a,b] &\to [0,1]\\
        y &\mapsto \frac{y-a}{b-a}.
    \end{split}
    \] 
\end{Ex}
\begin{Ex}
    Диск $\mathcal{D}^{n} = \left\{x\in \mathbb{R}^{n} \st \sum\limits_{i=1}^{n} x_i^2 \leqslant  1\right\}$ гомеоморфен полусфере \\$\mathcal{S}^n_+ =  \left\{x\in \mathbb{R}^{n+1} \st \sum\limits_{i=1}^{n} x_i^2 =  1  \; \land \; x_{n+1} \geqslant 0 \right\}$.

    Построим гомеоморфизм, который можно показать на рисунке (см. Рис. \ref{homeo_disk_halfsphere}), легко понять: нужно просто натянуть диск на полусферу, как кусок резины на поверхность шарика, и не менее просто записать явно.
    \image{homeo_disk_halfsphere}{Пояснительная картинка к построению гомеоморфизма между диском и полусферой}
    Так, прямое отображение 
    \[
    \begin{split}
        f: D^{n} &\to S^n_+\\
        (x_1, \dots, x_{n-1}) &\mapsto \left(x_1, \dots, x_{n-1}, \sqrt{1-\sum\limits_{i=1}^{n-1} x_i^2}\right).
    \end{split}
    \] 

    И обратное 
    \[
    \begin{split}
        g: S^n_+ &\to D^n\\
        (x_1, \dots, x_{n-1}, x_n) &\mapsto (x_1, \dots, x_{n-1}).
    \end{split}
    \] 
\end{Ex}
\begin{Ex}
    Диск $\mathcal{D}^{n} = \left\{x\in \mathbb{R}^{n} \st \sum\limits_{i=1}^{n} x_i^2 \leqslant  1\right\}$ гомеоморфен $\mathbb{R}^n$.

    Рассмотрим сначала простой случай $(-1,1) \homeo \mathbb{R}$. Заметим, что функция $f = \tg(\frac{\pi}{2}x)$ является гомеоморфизмом. 

    Тогда для общего случая можем использовать $f = \tg\left(\dfrac{\pi}{2} \|x\| \right) \frac{x}{\|x\|}$. Эта идея показана на Рис. \ref{homeo_disk_euclid}.
    \image{homeo_disk_euclid}{Поясненительная картинка к построению гомеоморфизма между диском и полусферой}
\end{Ex}
\begin{Ex}
    Сфера без точки $\mathcal{S}^{n} \setminus \{\mathrm{pt}\}$ гомеоморфна пространству $\mathbb{R}^{n}$.

    Построим гомеоморфизм, который можно показать на рисунке (см. Рис. \ref{homeo_sphere_euclid}). Такое отображение в литературе называют стереографической проекцией и задается как 
\begin{figure}[H]
    \centering
    \incfig[1]{homeo_sphere_euclid}
    \caption{}
    \label{homeo_sphere_euclid}
\end{figure}

    Прямое отображение
    \[
    \begin{split}
        f: \mathcal{S}^n\setminus \{\mathrm{pt}\} &\to \mathbb{R}^n\\
        (x_1, \dots, x_{n+1}) &\mapsto \left(\frac{x_1}{1-x_{n+1}}, \dots, \frac{x_n}{1-x_{n+1}} \right)
    \end{split}
    \] 

    И обратное
    \[
    \begin{split}
        g: \mathbb{R}^n &\to \mathcal{S}^n\setminus \{\mathrm{pt}\}\\
        (y_1, \dots, y_n) &\mapsto \left( \frac{2y_1}{1+\sum\limits_{i=1}^{n} y_i^2}, \dots, \frac{2y_n}{1+\sum\limits_{i=1}^{n} y_i^2}, \frac{-1+\sum\limits_{i=1}^{n} y_i^2}{1+\sum\limits_{i=1}^{n} y_i^2} \right)
    \end{split}
    \] 
\end{Ex}
\begin{Note}
    На самом деле, мы нигде явно не показали, что предложенные отображения непрерывны. На самом деле, так как все эти примеры построены в привычном нам пространстве $\mathbb{R}^n$, то можно ссылаться на непрерывности там. Где-то также могут понадобиться непрерывности нормы и проекции, что предлагается рассмотреть в качестве упражнения.
\end{Note}
    \begin{Task}
        Докажите следующие утвреждения (топология стандартная):
        \begin{enumerate}
            \item Любая норма $\| \cdot \|$, введенная на $\mathbb{R}^n$, является непрерывным отображением.
            \item Проекция $\pi\st \mathbb{R}^n \supset A \to B \subset \mathbb{R}^k$ является непрерывным отображением.
        \end{enumerate}
    \end{Task}

	\section{Аксиомы отделимости}
\begin{Intro}
    В данном параграфе речь пойдет о первом (для нас) топологическом инварианте, связанном с тем, как мы можем отделить какие-то множества (в простейших случаях точки) друг от друга.

    На самом деле, за время существования науки свойств, связанных с отделимостью, придумали достаточно много. Мы ограничимся наиболее важными для нас, но посмотреть их (возможно, неполный список) можно, к примеру, на \href{https://en.wikipedia.org/wiki/Separation_axiom#Main_definitions}{википедии}.
\end{Intro}

Забавно, что стандартные аксиомы называются <<T аксиомы>>. На самом деле корни у этого идут их специально придуманного немецкого слова Trennungsaxiom (разделение + аксиома). 

\subsection{Аксиома $T_0$, Колмогорова}
\begin{Def}
    [Аксиома $T_0$, Колмогорова]
    Топологическое пространство \topX удовлетворяет нулевой аксиоме отделимости (аксиоме Колмогорова или, проще говоря, является $T_0$ пространством), если для любых двух точек можно найти окрестность, содержащую одну точку и не содержащую другую.

    Переписывать данное определение символьно не очень удобно, но тоже можно
    \[
    \forall x, y \in \X \implies \exists U\open \X  \st  (x \in U \wedge y \notin U) \vee (y \in U \wedge x \notin U).
    \] 
\end{Def}

\begin{Ex}
    Практически любое привычное пространство является $T_0$.
\end{Ex}
При этом все-таки не совсем все пространства удовлетворяют аксиоме Колмогорова. В частности, в следующем примере показано, почему антидискретная топология не является таковой.
\begin{Ex}
    Антидискретная топология не является $T_0$ пространством.

    Действительно, в антидискретной топологии открытыми являются только пустое множество и само множество $X$. А значит это единственные окрестности, которые можно рассматривать. Но они всегда либо содержат обе точки (в случае всего множества), либо не содержат ни одной (в случае пустого множества). А значит определение не выполняется и такая топология не удовлетворяет аксиоме $T_0$.
\end{Ex}


\subsection{Аксиома $T_1$, Фреше}
\begin{Def}
    [Аксиома $T_1$, Фреше]
    Топологическое пространство \topX удовлетворяет первой аксиоме отделимости (аксиоме Фреше или, проще говоря, является $T_1$ пространством), если для каждой точки можно найти окрестность, не содержащую любую другую точку.

    Символьно можно записать это как
    \[
    \forall x\in \X \; \forall y\in \X \implies \exists U_x\open \X \st y \notin U.
    \] 
\end{Def}

Аксиомы отделимости постепенно будут добавлять какие-то свойства на пространство. Первым оказывается привычное нам свойство замкнутости точки.

\begin{Lem}
    Пространство является $T_1$-пространством тогда и только тогда, когда любое одноточечное множество является замкнутым. 
\end{Lem}
\begin{Proof}
    $\implies$ Рассмотрим произвольную точку $x\in X$. Для каждой точки $y_i \ne x$ найдем окрестность $U_{y_i}$ из определения $T_1$ пространства (то есть содержащую $y_i$ и не содержащую $x$ ). 

    Объединяя все такие окрестности, мы фактически пройдем по всем точкам кроме самой $x$, а значит
    \[
        \underset{\tiny \stackrel{y\in X}{y\ne x}}{\cup} U_{y} = X \setminus \{x\} \open \X.
    \] 
    А тогда $\{x\} \closed \X$ как дополнение к открытому.

    $\Longleftarrow$ Для любой точки пространства $x$ можем рассмотреть окрестность $U = X \setminus \{x\} \open \X$. Понятно, что эта окрестность отделяет любую другую точку от $x$.
\end{Proof}

\begin{Ex}
    Практически любое привычное пространство является $T_1$.
\end{Ex}

\begin{Ex}
    Так называемое связное двоеточие $\left(\{a,b\}, (\left\{ \emptyset, a, \{a,b\} \right\}\right)$ не является $T_1$ пространством.

    Нетрудно проверить, что дополнение к множеству $\{a\}$, равное $\{b\}$ не является открытым. А значит точка $\{a\}$ не является замкнутым множеством и все пространство не может быть $T_1$.
\end{Ex}

Следующая лемма (как и аналогичные ей далее) приведется без доказательства во многом, потому что она очевидна. Однако эта лемма показывает основную идею при построении новых аксиом отделимости. Грубо говоря, <<следующая по номеру аксиома наследует свойства предыдущего>>. Это как бы позволяет на практике перебирать их последовательно.
\begin{Lem}
    \[T_0 \subset T_1\]
\end{Lem}

\subsection{Аксиома $T_2$, Хаусдорфа}
\begin{Def}
    [Аксиома $T_2$, Хаусдорфа]
    Топологическое пространство \topX удовлетворяет второй аксиоме отделимости (аксиоме Хаусдорфа или, проще говоря, является $T_2$ пространством), если для любых двух точек можно найти непересекающиеся окрестности.

    Символьно можно записать это как
    \[
    \forall x,y \in \X  \implies \exists U_x, V_y\open \X \st U_x \cap V_y = \emptyset.
    \] 
\end{Def}

\begin{Ex}
    Метрическое пространство является Хаусдорфовым.

    Это нетрудно заметить, построив соответствующие окрестности радиусами, равными половине расстояния между точками.
\end{Ex}

\begin{Ex}
    Прямая Зариского $\left(\mathbb{R}, \left\{ \mathbb{R} \setminus F \st F \text{ -- конечное} \right\} \right)$ не является $T_2$ пространством.

    Понятно, что в такой топологии пересечение любых двух открытых множеств $U_x = \mathbb{R} \setminus F_1$ и $U_y = \mathbb{R} \setminus F_2$ (где $F_i$ -- конечные множества) есть 
   \[
       U_x \cap U_y = (\mathbb{R} \setminus F_1 ) \cap (\mathbb{R} \setminus F_2) = \mathbb{R} \setminus (F_1 \cup F_2) \open \mathbb{R}.
   \] 
   При этом $F_1 \cup F_2$ конечно, как объединение конечных, а значит $\mathbb{R} \setminus (F_1 \cup F_2)$ непусто. То есть нарушается условие Хаусдорфовости.
\end{Ex}

\begin{Lem}
    Пусть $f\st \X \to \Y$ -- непрерывное отображение, причем $\Y$ -- хаусдорфово. Тогда график этой функции $\Gamma_f = \left\{ (x, f(x) \st x\in \X \right\}$ замкнут в произведении $\X \times \Y$.
\end{Lem}
\begin{Proof}
    Рассмотрим произвольные точки $x \in X$ и $Y \ni y \ne f(x)$.

    В силу Хаусдорфовости $Y$ найдутся непересекающиеся окрестности $U, V$ точек $y$ и $f(x)$ соответственно. Но тогда из непрерывности отображения $ \inv{f}(V) \open \X$, то есть множество $\{(x,y) \st y \ne f(x)\}$ открыто в произведении. А значит дополнение к нему, то есть $\Gamma_f$ замкнуто. 
\end{Proof}

\begin{Cor}
     \topX -- Хаусдорфово тогда и только тогда, когда диагональ $\Delta_x = \left\{ (x, x) \st x \in \X \right\}$ замкнута в $\X \times \X$.
\end{Cor}
\begin{Proof}
    $\implies$ Возьмем в предыдущей лемме $f(x) = x$.

    $\Longleftarrow$ Пусть $X$ не хаусдорфово. Рассмотрим $H = X \times X \setminus \Delta_x$. Так как $\Delta_x \closed X \times X$, то $H \open X\times X$. Для любой точки $(x,y) \in H$ можно выбрать окрестность, полностью лежащую в $H$, как элемент базы (ведь открытое должно раскладываться в объединение элементов базы). А в базе топологии произведения элементы имеют вид $U \times V$. То есть $U \times V \subset H$ и $U \times V \cap \Delta_x = \emptyset$. А тогда и $U\cap V = \emptyset$. То есть для точек $x,y \in X$ нашли непересекающиеся окрестности, отделяющие их. Иначе говоря, $X$   -- хаусдорфово.
\end{Proof}

Более того, в хаусдорфовых пространствах уже предел последовательности оказывается единственным.
\begin{Prop}
    Если \topX является хаусдорфовым, то любая последовательность имеет ровно один предел.
\end{Prop}
\begin{Proof}
Пусть есть два различных предела $a_1, a_2$ последовательности $\{x_n\}$. Найдем из хаусдорфовости отделяющие их окрестности $U_{a_1} \cap U_{a_2} = \emptyset $. 
    
Так как $a_1$ -- предел, то $\exists N_1 \in \mathbb{N} \st \forall n> N_1 \implies x_n \in U_{a_1}$. Аналогично $\exists N_2 \in \mathbb{N} \st \forall n> N_2 \implies x_n \in U_{a_2}$. Но тогда для $n > \max(N_1, N_2) \quad x_n \in U_{a_1} \cap U_{a_2}$, что противоречит предположению о непересечении окрестностей.
\end{Proof}

Цепочка включений аксиом отделимости, очевидно, продолжается.
\begin{Lem}
    $T_1 \subset T_2$.
\end{Lem}

\subsection{Аксиома $T_3$}
\begin{Def}
    [Регулярное пространство $R$]
    Топологическое пространство \topX называется регулярным, если произвольное замкнутое множество можно отделить от не содержащейся в ней точки.
    \[
    \forall A \closed \X \forall x \notin A  \quad \exists U_A, U_x \open \X  \st U_a \cap U_X = \emptyset.
    \] 
\end{Def}

Недочетом регулярности является ее несоответствие логике выполнения включений последующих аксиом в предыдущие. Один из таких случаев показан в следующем примере.
\begin{Ex}
    Тривиальная топология является регулярной, но не Хаусдорфовой.

    Как мы знаем, тривиальная топология вообще не является даже $T_0$, а значит и $T_2$. Однако в ней есть ровно два замкнутых множества: пустое и все множество, которые легко отделяются от несодержащихся в них точках (в одном вообще нет точек, второе просто содержит все точки).
\end{Ex}

\begin{Note}
    Получается, что желаемую нами цепочку $T_0 \subset T_1 \subset T_2$ нельзя продолжить регулярностью. Именно поэтому ее не принято называть $T_3$. 

    Для выполнения соотношения  $T_2 \subset T_3$ добавляют (на самом деле в разной литературе по-разному) дополнительные условия.
\end{Note}

\begin{Def}
    [Аксиома $T_3$]
    Топологическое пространство \topX удовлетворяет третьей аксиоме отделимости (является $T_3$ пространством), если оно является регулярным и также удовлетворяет условию  $T_1$.
\end{Def}

Теперь, когда все точки замкнуты, их легко отделить непересекающимися окрестностями, что дает хаусдорфовость. Более строгое утверждение, продолжающее нашу цепочку, записано в лемме.
\begin{Lem}
\[T_2 \subset T_3\]
\end{Lem}

\subsection{Аксиома $T_4$, нормальность}
\begin{Def}
    [Нормальность]
    Топологическое пространство \topX является нормальным, если можно отделить любые два непересекающихся замкнутых множества.
    \[
        \forall F_1, F_2 \closed \X \; \exists U_{F_1}, U_{F_2} \open \X \st U_{F_1} \cap U_{F_2} = \emptyset.
    \] 
\end{Def}

\begin{Ex}
    Любое метрическое пространство является нормальным.

    Действительно, давайте для произвольных замкнутых множеств $F_1, F_2$ рассмотрим такую функцию $f\st X \to \mathbb{R}$
    \[
        f(x) = \frac{\rho(x, F_1)}{\rho(x, F_1) + \rho(x, F_2)},
    \] где под $\rho(x, A)$ понимается расстояние от точки до множества, которое задается как $\inf\limits_{a\in A} \rho(x,a)$.

    Тогда, в силу непрерывности метрики и арифметических операций, $f$ -- непрерывна. А значит можем рассмотреть в качестве окрестностей, отделяющих $F_1$ от $F_2$ 
    \[
        U_{F_1} = \inv{f}((-\infty, \tfrac12)),\quad U_{F_2} = \inv{f}((\tfrac12, \infty)).
    \] 
    (они открыты, как прообразы открытых; а непересекаемость можно проверить, посмотрев на то, какие значения выдает функция при разных $x$ )
\end{Ex}

И снова оказывается, что нормальность не продолжает цепочку включений $T_{i-1} \subset T_i$. Поэтому, аналогично предыдущему, можно добавить первую аксиому. Оказывается, что этого достаточно.
\begin{Def}
    [Аксиома $T_4$]
    Топологическое пространство \topX удовлетворяет четвертой аксиоме отделимости (является $T_4$ пространством), если оно является нормальным и также удовлетворяет условию  $T_1$.
\end{Def}
    
    Оказывается, что $T_4$ пространства позволяют не только отделить замкнутые друг от друга, но также ввести функциональную отделимость. Это примерно то, что было представлено для доказательства нормальности метризуемых пространств. Только теперь верно для любых $T_4$ пространств.
    \begin{Lem}
        [Урысона]
        В $T_4$ пространствах для любых непересекающихся замкнутых множеств $F_1, F_2$ существует отображение  $f \st X \to I$, что $f(F_1) = \{0\}$ и $f(F_2) = \{1\}$.
        \[
            \forall F_1, F_2 \closed X \st F_1 \cap F_2 = \emptyset \implies \exists f \st X \to I \st f(F_1) = \{0\} \text{ и } f(F_2) = \{1\}.
        \] 
    \end{Lem}
    Доказательство данной леммы оказывается не столь очевидным и требующим неких матанализных выкладок, поэтому не будем его приводить.

    Как уже говорилось, для введенного таким образом $T_4$ пространства включения аксиом продолжаются.
    \begin{Lem}
        \[
        T_3 \subset T_4
        \] 
    \end{Lem}

	\section{Аксиомы счетности}
\begin{Intro}
    В данном параграфе мы рассмотрим следующий набор свойств топологических пространств, которые позволяют их отличать. Это, так называемые, аксиомы счетности. Их основой является желание ответить на вопрос, можем ли мы чем-то ограничить какую-то топологическую структуру сверху. То есть требуется понять, будет ли нечто счетным. А вот что стоит за этим \textit{нечто}, мы и посмотрим. Всего мы рассмотрим три ограничения. Два из них пронумерованы, а вот третье --- как бы немного обособленное и без номера.
\end{Intro}
\subsection{1 аксиома счетности}
\begin{Def}[1 аксиома счетности]
    Говорят, что топологическое пространство \topX удовлетворяет первой аксиоме счетности, если для любой точки $x\in X$ есть не более чем счетная база $\mathcal{B} = \{ V_i \}_{i\in I}$ окрестностей точки  $x$, то есть в каждой окрестности $U_x$ содержится некоторая окрестность из базы. 
\end{Def}

\begin{Prop}
    Любое метризуемое пространство удовлетворяет 1 аксиоме счетности.
\end{Prop}
\begin{Proof}
    Рассмотрим в качестве счетной базы окрестностей точки $x$ семейство шаров с центром в точке $x$ радиусов $\frac{1}{n}$, где $ n\in \mathbb{N}$. 
\end{Proof}

\begin{Ex}
    Топология Зариского не удовлетворяет 1 аксиоме счетности.

    Действительно, если существует база $\mathcal{B} = \{ U_k \}_{k=1}^\infty$ окрестностей точки $x$, то рассмотрим точку $y$ из множества $\mathbb{R} \setminus \left( \{x\} \cup \bigcup\limits_{k=1}^{\infty} (\mathbb{R} \setminus U_k) \right)$. 

    Так как одноточечное множество замкнуто, то $\mathbb{R} \setminus \{y\} \open X$. Но эта окрестность не содержит в себе ни одной из окрестностей $\mathcal{B}$, так как ровно их мы и вычитаем. А значит $\mathcal{B}$ не является базой. То есть пришли к противоречию, а значит никакой счетной базы окрестностей точки быть не может.
\end{Ex}

\subsection{2 аксиома счетности}
\begin{Def}
    [2 аксиома счетности]
    Говорят, что топологическое пространство \topX удовлетворяет второй аксиоме счетности, если у него есть не более чем счетная база.
\end{Def}

\begin{Lem}
    Из 2 аксиомы счетности следует 1 аксиома счетности.
\end{Lem}
\begin{Proof}
    Если есть некоторая счетная база у всего пространства, то можно ее же рассматривать как счетную базу окрестностей точки. 
\end{Proof}

    
\begin{Th}
    [Линделефа]
    Если пространство \topX удовлетворяет 2 аксиоме счетности, то из любого покрытия $X$ можно выбрать не более чем счетное его подпокрытие.
\end{Th}
\begin{Proof}
    Пусть $\mathcal{B} = \{V_k\}_{k\in\mathbb{N}}$ -- счетная база из 2 аксиомы счетности, и $\{U_i\}_{i\in I}$ -- некоторое покрытие $X $.

    Для каждого $V_k$ выберем такое множество $U_{i(k)}$ из покрытия, что $V_k \subset U_{i(k)}$. 

    Тогда $X = \bigcup\limits_{k \in \mathbb{N}} V_k \subset \bigcup\limits_{k \in \mathbb{N}} U_{i(k)}$. При этом также $\bigcup\limits_{k \in \mathbb{N}} U_{i(k)} \subset X $. А значит на самом деле они просто равны, то есть $\{ U_{i(k)} \}_{k\in \mathbb{N}}$ -- необходимое подпокрытие.
\end{Proof}

\begin{Ex}
    Прямая Зоргенфрея не удовлетворяет 2 аксиоме счетности.

    Действительно, открытое множество $[x, x+1)$ для каждой точки должно содержать некоторый элемент базы. Однако для различных точек  $x\neq y$ эти элементы базы отличаются, так как для $x<y \quad x \notin [y, y+1)$ . То есть элементов базы по крайней мере столько же, сколько и точек на прямой, а это точно не счетно.
\end{Ex}

\subsection{Сепарабельность}
\begin{Def}
    [Сепарабельное пространство] 
    Говорят, что топологическое пространство \topX сепарабельно, если у него есть не более чем счетное всюду плотное подмножество.
\end{Def}

\begin{Lem}
    Из 2 аксиомы счетности следует сепарабельность.

    При этом, если пространство метризуемо, то верно и обратное.
\end{Lem}
\begin{Proof}
    Рассмотрим некоторую счетную базу $\mathbb{B} = \{B_i\}_{n=1}^\infty$ и выберем из каждого элемента этой базы по одной точке. Полученное множество назовем $H$ (то есть $H = \{ x_i \st x_i \in B_i\}$). 

    Во-первых, заметим, что $H$ счетно. А во-вторых, оно всюду плотно, так как любая другая окрестность точно содержит в себе элемент базы, а значит и соответствующую  точку $x_i$.

    В случае же метризуемого пространства, для всюду плотного $A\subset X$ рассмотрим множество $\{B_r(x) \st x \in A, \; r \in \mathbb{Q}\}$. Оно счетное, как счетное объединение счетных множеств. Утверждается, что это множество является базой. Для доказательства этого достаточно лишь показать, что для любой точки $x$ открытого множества $U$ (в данном случае достаточно сказать лишь про шары)  есть элемент этого множества, лежащий в $U$ и содержащий $x$.

    Рассмотрим точку $p\in X$ и некоторый шар $B_{\varepsilon}(p)$. В силу плотности $A$ можно найти точку $q\in A$, что $\rho(p,q) < \frac{\varepsilon}{3}$. В силу плотность рациональных чисел между $\frac{\varepsilon}{3}$ и $\frac{2\varepsilon}{3}$ можно найти рациональное число $\delta$. А тогда шар $B_\delta (q)$ содержится в $B_\varepsilon(p)$ и содержит $p$, что и требовалось.
\end{Proof}

Заметим, что данная лемма позволяет нам сказать о неметризуемости прямой Зоргенфрея. Так как в обратном случае сепарабельность с метризуемостью дали бы 2 аксиому счетности. А мы ранее убедились, что она не удовлетворяет 2 аксиоме счетности.

	\section{Компактность}
\begin{Intro}
 Понятие компактности является одним из основных в топологии и используется повсеместно в ее приложениях. На самом деле это некоторое подобие конечности в множествах. Также появляется более мягкое понятие паракомпактности, которое весьма понадобится при изучении многообразий.
\end{Intro}

\subsection{Компактное пространство}
Понятие компакта тесно завязано на покрытиях. Интуитивно понятно, что это такой набор, который \textit{покрывает} полностью исходное множество. Рассмотрим более строгие определения.
\begin{Def}
     [Покрытие]
     Множество $\{S_i\}_{i\in I}$ открытых подмножеств пространства \topX называется покрытием этого пространства, если $\X \subset \underset{\tiny i\in\mathcal{I}}{\cup} S_i$.
\end{Def}
\begin{Def}
    [Подпокрытие]
    Множество $\tilde{S}$ называется подпокрытием для покрытия $S$ пространства \topX, если $\tilde{S}$ -- покрытие для $\X$, и $\tilde{S} \subset S$.
\end{Def}
\begin{Def}
    [Компактное пространство]
    Пространство \topX называется компактным, если из любого его покрытия можно выбрать конечное подпокрытие этого пространства.
\end{Def}
\begin{Note}
    Заметим, что скорее всего Вы уже встречались с понятием компакта в математическом анализе. Там появлялась следующая лемма, показывающая какие множества являются компактными в $\mathbb{R}^n$. Ее мы приведем здесь без доказательства.
\end{Note}
\begin{Lem}
    Пространство $\X \subset \mathbb{R}^n$ является компактным тогда и только тогда, когда  $\X$ замкнуто и ограничено.
\end{Lem}

Иногда вывод о компактности какого-то множества можно сделать не по определению, а пользуясь набором следующих свойств.
\begin{Lem}
    [Свойства компакта]
    Пусть $\X, \Y$ -- компакты, $f\st \X \to \Y$ -- непрерывное.

    Тогда
    \begin{enumerate}
        \item $\X \times \Y$ -- компакт,
        \item Образ функции $f(X)$ -- компакт,
        \item Произвольная функция  $g\st \X \to \mathbb{R}$ достигает своего максимума и минимума,
        \item Замкнутое подмножество $A \closed \X$ является компактом,
        \item Компактное подмножество $A$ Хаусдорфового пространства $\X$ является замкнутым $A \closed \X$.
    \end{enumerate}
\end{Lem}
\begin{Proof}
    \begin{enumerate}
        \item TODO
    \item Рассмотрим некоторое покрытие $f(X) = \cup_{i=1}^\infty U_i$. В силу непрерывности $\cup_{i=1}^\infty \inv{f}(U_i) = X$. При этом $X$ -- компакт, а значит можно выбрать его конечное подпокрытие $\mathcal{U} = \{ U_j \}_{j=1}^{N}$. 

        А $f(\mathcal{U} )$ покрывает образ отображения, что нам и требовалось.
    \item По предыдущему пункту  $f(X)$ -- компакт. Однако единственный компакт в $\mathbb{R}$ -- отрезок. А на нем достигается и минимум и максимум.
    \item TODO
    \item TODO
    \end{enumerate}
\end{Proof}

Очень важной является следующая теорема, позволяющая делать выводы о гомеоморфизме некоторой функции (собственно о том, что мы глобально и хотим исследовать).
\begin{Th}
    Непрерывная биекция $f\st \X \to \Y$ из компакта в Хаусдорфово пространство является гомеоморфизмом.
\end{Th}
\begin{Proof}
    Понятно, что нам достаточно доказать непрерывность обратного $\inv{f}$, то есть 
    \[
        \forall Z \closed X \implies \inv{(\inv{f})} (Z) = f(Z)  \closed Y.
    \] 
    Из 4-ого свойства такое $Z$ является компактом. А тогда из 2-ого свойства и $f(Z)$ --- компакт. И, таким образом, из 5-ого свойства получаем замкнутость $f(Z) \closed Y$.
\end{Proof}

\subsection{Секвенциальный компакт}
В некоторых случаях оказывается удобнее оперировать последовательностями, а не покрытиями. Тогда можно говорить о секвенциальном компакте. Как мы увидим, в привычных нам пространствах это понятие равносильно обычной компактности.
\begin{Def}
    [Секвенциальный компакт]
    Пространство \topX является секвенциально компактным, если из любой последовательности можно выбрать сходящуюся подпоследовательность.
\end{Def}
\begin{Th}
    Если пространство удовлетворяет второй аксиоме счетности, то секвенциальная компактность равносильна компактности.
\end{Th}

\subsection{Паракомпактность}
К сожалению, далеко не все пространства компактны. Но хочется иметь какую-то более слабую альтернативу. Одной из таких важных альтернатив является паракомпактность, с которой связано часто применяющееся понятие разбиения единицы.
\begin{Def}
    [Локальная компактность]
    Пространство \topX называется локально компактным, если около любой точки можно найти компактную окрестность.
\end{Def}
\begin{Def}
    [Локально конечное покрытие]
    Покрытие называется локально конечным, если вокруг любой точки можно найти окрестность, пересекающую конечное число множеств из этого покрытия.
\end{Def}

\begin{Def}
    [Паракомпактное пространство]
    пространство \topX называется паракомпактным, если в любое покрытие можно вписать локально конечное покрытие.
\end{Def}

\begin{Ex}
    Стандартная топология над $\mathbb{R}^n$ является паракомпактной.
\end{Ex}

\begin{Prop}
    Из компактности следует паракомпактность.
\end{Prop}

\section{Разбиение единицы}
Для формулирования понятия разбиения единицы нам понадобится носитель функции --- что-то близкое к множеству, на котором функция не обнуляется.
\begin{Def}
    [Носитель функции]
    Носителем $\mathrm{supp}\,f$ функции $f\st X \to \mathbb{R}$ называется множество $\mathrm{Cl} \{x\in X \st f(x) \neq 0\}$.
\end{Def}

\begin{Def}
    [Разбиение единицы]
    Семейство неотрицательных функций $f_\alpha\st X \to \mathbb{R}_+$ называется разбиением единицы, если
    \begin{enumerate}
        \item $\mathrm{supp}\, f_\alpha$ составляют локально конечное покрытие $X$,
        \item $\sum\limits_{\alpha} f_\alpha(x) = 1$.
    \end{enumerate}
\end{Def}

	\section{Связность}
\begin{Intro}
    
\end{Intro}

\subsection{Связность}
\begin{Def}
    [Связность]
    Топологическое пространство \topX называется связным, если его нельзя представить как объединение двух открытых непересекающихся множеств.
    \[
    \forall U, V \open \X \st U\cap V = \emptyset \implies X \neq U \cup V.
    \] 
\end{Def}

\begin{Note}
    Также можно построить определение через замкнутые множества
    \[
    \forall F, H \closed \X \st F\cap H = \emptyset \implies X \neq F \cup H.
    \] 
\end{Note}

\begin{Prop}
    Образ непрерывного отображения связного множества является связным.
    \[
        f\st \X \to \Y \text{ -- непрерывное, } \X \text{ -- связное} \implies f(\X) \text{ -- связное}.
    \] 
\end{Prop}

\begin{Ex}
    Понятно, что любое выпуклое подмножество $\mathbb{R}^n$ является связным.
\end{Ex}

\begin{Lem}
    Топологическое пространство \topX связное тогда и только тогда, когда подмножествами, открытыми и замкнутыми одновременно, являются только $\emptyset$ и $\X$.
\end{Lem}

\begin{Ex}
    $[1,2] \cup [3,4]$ несвязное.
\end{Ex}
\begin{Note}
    Заметим, что в последнем примере множество состояло из двух связных множеств: $[1,2]$ и $[3,4]$ -- но не являлось связным. Это подводит к идее рассматривать отдельно связные части всего множества. Такие множества будем называть компонентами связности.
\end{Note}
\subsection{Компоненты связности}
\begin{Def}
    [Компоненты связности]
    Компонентой связности назовем наибольшее по включению связное подмножество $ \X$.
\end{Def}

\begin{Note}
    В связи с введенным определением возникает небольшой вопрос об его полезности. Интуитивно хочется, чтобы всегда можно рассматривать пространство как набор компонент связности. А уже с каждой из них работать по отдельности. Этот вопрос закрывает следующее утверждение.
\end{Note}

\begin{Prop}
    Любое пространство можно разбить на связные компоненты связности, то есть
    \begin{enumerate}
        \item каждая точка содержится ровно в одной компоненте связности,
        \item различные компоненты связности не пересекаются.
    \end{enumerate}
\end{Prop}

\begin{Lem}
    Замыкание связного множества является связным.
    \[
        A \text{ -- связное} \implies \mathrm{Cl}(A) \text{ -- связное}.
    \] 
\end{Lem}

\begin{Cor}
    Компоненты связности замкнуты.
\end{Cor}

\subsection{Линейная связность}
\begin{Def}
    [Путь]
    Путем, начинающимся в точке $x$ и заканчивающемся в точке $y$ пространства $\X$, назовем непрерывное отображение $\alpha \st [0,1] \to \X$, что $\alpha(0) = x$ и $\alpha(1) = y$.
\end{Def}
\begin{Def}
    [Линейная связность]
    Пространство \topX называется линейно связным, если между двумя любыми точками существует путь.
    \[
        \forall x, y \in \X \implies \exists \alpha\st [0,1] \to \X \text{ -- непрерывное} \st \alpha(0) = x, \, \alpha(1) = y.
    \] 
\end{Def}

\begin{Prop}
    Любое линейно связное пространство является связным.
\end{Prop}

\begin{Prop}
    Образ непрерывного отображения линейно связного множества является линейно связным.
\end{Prop}

\begin{Note}
    Заметим, что обратное неверно! В качестве примера рассмотрим, так называемую, топологическую синусоиду.

    \[
        \mathcal{S} = \left\{ \sin \tfrac1x \st x \in (0, 1]  \right\} \cup \left\{ (0,0) \right\}.
    \] 
\end{Note}
\begin{Prop}
    Топологическая синусоида является связной, но не является линейно связной.
\end{Prop}

\begin{Note}
    Естественно, по аналогии с обычной связностью для линейно связных пространств тоже можно рассматривать компоненты связности (только теперь уже линейные компоненты связности). 
\end{Note}
 \begin{Prop}
   Отношение соединения путем двух элементов является отношением эквивалентности.

   В частности, классами эквивалентности являются линейные компоненты связности.
\end{Prop}

\subsection{Локальная линейная связность}
ПОКА НЕ ЗНАЮ, НАСКОЛЬКО НУЖНО.

	\section{Факторпространства}
\begin{Intro}
    
\end{Intro}

\subsection{Определение и свойства}
\begin{Def}
    [Фактортопология]
    Пусть есть топологическое пространство \topX и на множестве $\X$ введено отношение эквивалентности $\sim$. Тогда на множестве (или, как его еще называют, фактормножестве) $\X / \sim$ можно построить топологию:
    \[
        U \open \X / \mathord{\sim} \Leftrightarrow \inv{\pi}(U) \open \X,
    \]где $\pi \st \X \to \X / \sim$
\end{Def}

\begin{Task}
    В том, что определение дано корректно, Вам предлагается убедиться самим.
\end{Task}

\begin{Prop}
    Фактортопология является тончайшей среди всех топологий с непрерывным отображением $\pi$. То есть $\Omega_{\text{остальные}} \subset \Omega_{\X / \sim}$
\end{Prop}

\begin{Lem}
    Фактортопология наследует свойства исходной топологии (компактность, связности, счетности\dots).
\end{Lem}


\begin{Lem}
Пусть дано отображение $f\st \X \to \Y$, что  $f(x) = f(y)$ для  $x\sim y$.    

Тогда существует единственное отображение $\bar{f}\st \X / \sim \to \Y$, причем непрерывность $f$ равносильна непрерывности $\bar{f}$.
\end{Lem}

\begin{Def}
    [Экстраполяция]
    Пусть $f\st \X \twoheadrightarrow \Y$ -- сюръекция, причем  $\X$ -- компакт, а  $\Y$ -- Хаусдорфово.

    Тогда $\bar{f}\st \X / \sim \to \Y$ -- гомеоморфизм. Это свойство называется экстраполяцией. 
\end{Def}

\begin{Ex}
    Рассмотрим отображение $f\st [0,1] \to \mathcal{S}^1$, заданное как $x \overset{f}{\mapsto} (\cos 2\pi x, \sin 2\pi x)$.
    
    Понятно, что оно сюръективно, но не биективно. А значит не может быть и гомеоморфизма.

    Однако, рассматривая фактор $[0,1] / \{0,1\}$ (стягиваем точки 0 и 1 в одну), получим, пользуясь экстраполяцией,
     \[
         [0,1] \overset{\pi}{\to} [0,1] / \{0,1\} \overset{\bar{f}}{\to} \mathcal{S}^1.
    \] 
    И при этом $\bar{f}$ -- гомеоморфизм.
\end{Ex}

\subsection{Конструкции на основе фактортопологии}
\begin{Def}
    [Стягивание]
    Стягиванием $\X / A$ пространства  $\X$ по множеству $A \subset \X$ назовем факторпространство, получаемое из отношения $x\sim y \Leftrightarrow (x=y \text{ или } (x\in A \text{ и } y \in A))$. 
\end{Def}
\begin{Ex}
    Стянем сферу по какой-нибудь из окружностей, секущих ее. Получим две сферы, касающиеся в точке.
\end{Ex}

\begin{Def}
    [Приклеивание]
    Пусть $f\st \X \to  \Y$. Приклеиванием $\X \underset{f}{\sqcup} \Y$ назовем пространство $\X \times [0,1] \sqcup \Y / (x,1) \sim f(x)$.
\end{Def}
\begin{Ex}
    НЕ ПРИДУМАЛ ПОКА
\end{Ex}
\begin{Def}
    [Конус]
    Конусом $C\X$ над пространством $\X$ назовем $\X \times [0,1] / (x,1) \sim \{\mathrm{pt}\}$.
\end{Def}
\begin{Ex}
    Конус над окружностью --- это привычный нам конус.
\end{Ex}

\begin{Def}
    [Надстройка]
    Надстройкой $\Sigma \X$ над пространством $\X$ назовем пространство $C\X / \X \times \{0\}$.
\end{Def}

\begin{Ex}
    Надстройка над окружностью дает <<двусторонний конус>>.
\end{Ex}

	\section{Некоторые топологические конструкции}
\subsection{Одноточечная компактификация}
\subsection{Проективное пространство}


	\section{Гомотопии}
\begin{Intro}
В этом параграфе мы будем пытаться перенести конструкции из топологии в элементы линейной алгебры. Оказывается, в топологическом пространстве есть (в основном) группы, которые к тому же сохраняются при гомеоморфизме. Первыми представителями таких групп являются гомотопические группы. Однако для их построения нам необходимо погрузиться в понятие гомотопии, позволяющее разделять отображения на (в каком-то смысле) одинаковые и различные. Описывая интуитивно, можно говорить о непрерывных деформациях отображений. Прочувствовать эту идею легко на двух путях между двумя точками: они будут одинаковыми, если мы сможем постепенно, сантиметр за сантиметром, сдвигать один путь в направлении другого. Но если между этими путями будет болото, то мы не сможем так сделать.

Вообще, теория гомотопий и гомотопических групп отделяется в целую отдельную науку. Мы лишь коснемся самых основ и плавно подойдем к простейшей из них --- фундаментальной группе, рассматриваемой в следующем параграфе.
\end{Intro}
\subsection{Пространства непрерывных отображений}
Однако для начала нужно понять, над каким пространством мы будем издеваться и кого будем деформировать.

Будем рассматривать теперь всевозможные непрерывные отображения между топологическими пространствами $X$ и $Y$. Такое множество будем называть $\mathcal{C}(X, Y)$. Возникает вопрос: можно ли ввести какую-то топологию на этом множестве?

Конечно, всегда существуют дискретная и антидискретная топологии. Но они не несут большого практического применения.

В качестве первого варианта рассмотрим метрическую топологию, которую можно получить, если $Y$ является метризуемым. 
\begin{Def}
    [Топология равномерной сходимости]
    Топология на $\mathcal{C}(X,Y)$, индуцированная метрикой $\mu(f_1, f_2) = \underset{\tiny x \in X}{\sup} \{\rho(f_1(x), f_2(x))\}$, где $\rho$ --- метрика в топологии $Y$, называется топологией равномерной сходимости.
\end{Def}

Другим, отчасти противоположным, примером топологии на непрерывных отображениях, является топология поточечной сходимости. 
\begin{Def}
    [Топология поточечной сходимости]
    Топология на $\mathcal{C}(X,Y)$, базой которой является множество из всевозможных пересечений множеств  $\{f \in \mathcal{C}(X,Y) \st f(x_i) \in U_i\}$, где $x_1, \dots, x_k \in X$, $U_1, \dots U_k \open Y$.
\end{Def}

Данные названия топологий говорят о свойствах внутри них. Но анализом различных сходимостей все-таки занимается функциональный анализ. Поэтому построим топологию таким образом, чтобы воспользоваться топологическими свойствами.

\begin{Def}
    [Компактно-открытая топология]
    Топология на $\mathcal{C}(X,Y)$, базой которой является множество из всевозможных пересечений множеств  $\{f \in \mathcal{C}(X,Y) \st f(K) \subset U\}$, где $K \subset X$ -- компакт, $U \open Y$.
\end{Def}

\begin{Note}
    В таким образом введенной топологии, в частности, можно брать отрезки в качестве компакта. И получившиеся отображения будут являться путями в пространстве $Y$. Нам практически понадобится именно эта составляющая в ближайших параграфах.
\end{Note}
\subsection{Определение гомотопии и примеры}
Слово гомотопия берет свои корни от греческих слов: homós -- одинаковые, tópos -- место. То есть гомотопия пытается описать одинаковые куски пространства. Вспоминая, что речь в данном параграфе идет про пространство $\mathcal{C}(X, Y)$, можно интуитивно переформулировать это понятие в одинаковость каких-то мест, связанных с образами отображений. Чуть более точно, сформулируем в следующем определении.
\begin{Def}
    [Гомотопия]
    Гомотопией между двумя отображениями $f,g \in \mathcal{C}(X,Y)$ назовем такое непрерывное отображение $H\st X\times I \to Y$, для которого $\forall x\in X \quad H(x, 0) = f(x)$, $H(x,1) = g(x)$. 
\end{Def}

Данное определение можно воспринимать как некоторая деформация между двумя отображениями: в начальный момент времени мы имеем функцию $f$, а в конце --- $g$. Все, что между, нас не очень интересует, кроме того, что там деформация непрерывна.

\begin{Ex}
    Простейшим примером является гомотопия для отображений $f_1, f_2 \in C(\{\mathrm{pt}\}, Y)$. На самом деле каждое из таких отображений лишь выбирает точку в пространстве $Y$, а гомотопия --- это путь между выбранными точками. 
\end{Ex}

 \begin{Note}
     В частности, можно увидеть, что является гомотопией двух путей, закрепленных между двумя точками $x_0$ и $x_1$. 

    По определению, для путей $\alpha, \beta \st I \to Y$ это $H\st I \times I \to Y$, для которого $H(x, 0) = \alpha(x), H(x,1) = \beta(x)$. При этом в силу непрерывности $H(0, t) = x_0, H(1, t) = x_1$, то есть все деформации имеют одно начало и конец.

    Из этого можно легко представить случаи гомотопных путей (Рис. \ref{fig:path_nohomo}) и негомотопным (Рис. \ref{fig:path_homo}). Понятно, что во втором случае непрерывной деформации мешает вырезанный кусок между этими двумя путями. 
\end{Note}

Понятно, что можно говорить вместо $F\st X\times I \to Y$ о параметризованном отображении $F_t \st X \to Y \quad t \in I$. Иногда таким образом оказывается удобнее описывать гомотопии.

\begin{Ex}
    В каком-то смысле \textit{удачным} пространством является выпуклое пространство (то есть то, в котором вместе с двумя точками содержится отрезок между ними). В нем любые два отображения можно соединить, так называемой, линейной гомотопией:
    \[
    H(x,t) = f(x) \cdot t + g(x) \cdot (1-t).
    \] 
    Нетрудно убедиться, что это действительно гомотопия.
\end{Ex}
    
Так как в самом слове гомотопия есть корень homós, означающий одинаковый, то появляется идея, что все гомотопные отображения в каком-то смысле одинаковые. А негомотопные --- отличающиеся. Конечно, как и всегда, строго это формулируется в утверждении про отношение эквивалентности.
 
\begin{Prop}
    Гомотопия является отношением эквивалентности на множестве $\mathcal{C}(X,Y)$. 

    Классы эквивалентности называют гомотопическими классами и обозначают $\pi(X,Y)$.
\end{Prop}
\begin{Task}
    Проверьте выполнение рефлексивности, симметричности и транзитивности.
\end{Task}

\begin{Ex}
    Как мы уже выяснили, в выпуклом пространстве все отображения гомотопны друг другу. А значит там есть ровно один гомотопический класс.
\end{Ex}

Говоря о главной теме всей главы --- гомеоморфизмах, можно снова вспомнить, что вообще-то это очень грубое определение. Для него обратное отображение должно давать вместе с прямым ровно тождественное отображение. Однако если предположить, что для пути с одинаковыми началом и концом (петли) обратным является просто проход в обратную сторону, то даже в таком тривиальном случае композиция с обратным не будет равна тождественному. Однако она точно будет гомотопна ему. Отсюда появляется идея ослабления требований гомеоморфизма.
\begin{Def}
    [Гомотопическая эквивалентность]
    $f\in \mathcal{C}(X,Y)$ называется гомотопической эквивалентностью между $X$ и $Y$, если существует $g\in C(Y,X)$ такое, что  $f\circ g \sim 1_Y$ и $f\circ f \sim 1_X$. 

    Пространства, между которыми можно построить гомотопическую эквивалентность будем называть гомотопически эквивалентными. 
\end{Def}


Гомотопическая эквивалентность чем-то отдаленно напоминает гомеоморфизм: есть непрерывное туда и обратно отображение. Однако в гомеоморфизме обратное отображение --- привычное нам, дающее в композиции с прямым тождественное; а в гомотопическом случае под обратным понимается более общее: такое, что их композиция с прямым лишь гомотопична тождественному. 

Нетрудно понять, что гомотопическая эквивалентность является обобщением гомеоморфизма. Ведь гомотопичность включает в себя равенство. Обратное же, очевидно, неверно, что легко увидеть в случае с путями с началом и концом в одной точке (так называемыми петлями). Складывая петлю с такой же, только пройденной в обратном направлении, мы, конечно, не получим тождественное отображение. Но полученное отображение точно гомотопично тождественному.

Некоторым простейшим частным случаем пространств являются стягиваемые пространства.
\begin{Def}
    [Стягиваемые пространства]
    Пространство $X$ называется стягиваемым, если $\exists x_0 \in X \quad 1_X \sim \mathrm{const}_{x_0}$  (то есть константное отображение совпадает с тождественным).
\end{Def}
На самом деле, по-настоящему, название стягиваемое раскрывается в следующей лемме.
\begin{Lem}
    $X$ --- стягиваемое тогда и только тогда, когда $\exists x_0\in X \st X \sim \{x_0\}$. 
\end{Lem}
\begin{Proof}
    $\implies\quad$ Пространства гомотопически эквивалентны, если есть эквивалентность между ними. В нашем случае подойдет $\mathrm{const}_{x_0}$, так как $\mathrm{const}_{x_0} \circ 1_X \sim 1_X$ и $1_X \circ \mathrm{const}_{x_0} \sim 1_X$ из определения стягиваемого пространства.

    $\Longleftarrow\quad$ Рассмотрим гомотопическую эквивалентность $f\st X \to \{x_0\}$ (которая есть в силу гомотопичности $X\sim \{x_0\}$). Понятно, что это может быть только константное отображение, просто потому что других вариантов отображения в одноточечное пространство нет. То есть  $\exists g \in C(\{x_0\} \to X) \st \mathrm{const}_{x_0} \circ g \sim 1_{\{x_0\}} \text{ и } g \circ \mathrm{const}_{x_0} \sim 1_X$. На самом деле же $g$ просто выбирает точку в $X$. Мы положим $g(x_0) = x_0$. Тогда $g\circ \mathrm{const}_{x_0} = x_0 = \mathrm{const}_{x_0}$ и получаем, что $\mathrm{const}_{x_0} \sim 1_X$.
\end{Proof}

То есть стягиваемость надо понимать именно в смысле этого слова: пространство можно стянуть (сжать, сдавить) в точку. 

\begin{Ex}
    Очевидным примером стягиваемого пространства является любое выпуклое евклидово пространство. В нем есть линейная гомотопия.

    В частности, легко представить себе стягивание шара в его центр. 
\end{Ex}
\subsection{Ретракции}
Вообще, мы уже много произносим слово <<деформация>>, всегда представляя произвольные действия над объектом (пространством). Намного логичнее может казаться оставлять уже стоящую в нужном положении часть на месте. Ну или хотя бы <<гомотопно на месте>>. Эти идеи проявляются в различных понятиях ретрактов.

\begin{Def}[Ретракт, Ретракция]
    Подпространство $A\subset X$ --- ретракт пространства $\X$, если существует такое непрерывное отображение $r\st X \to X$ (называемое ретракцией $\X$), что
    \begin{itemize}
        \item $r(X) = A$, 
        \item $\left. r \right|_A = \mathrm{id}$. 
    \end{itemize}
\end{Def}

\begin{Note}
    Иначе говоря, мы "сжимаем" пространство $\X$ до $A$, требуя, чтобы все точки  $A$ оставались на своих местах.

    Ретракцию можно удобно изобразить на коммутативной диаграмме TODO \ref{diag:retract}. 
\end{Note}

Оказывается, что такое определение хоть и весьма полезно (что мы увидим дальше в одном из предложений), но может быть усилено почти до аналогии на гомотопические эквивалентности.

\begin{Def}
    [Деформационный ретракт]
    Подпространство $A \subset X$ --- деформационный ретракт пространства $\X$, если существует ретракция $r$, гомотопная тождественному отображению на $X$ ($r \sim \mathrm{id}_X$).

    Если к тому же в точках $a\in A$ отображение совпадает с  $\mathrm{id}_X$ (то есть $a$ под действием гомотопии остаются неподвижны), то $A$ называется строгим деформационным ретрактом.
\end{Def}
 
\begin{Ex}
    Понятно, что всякое одноточечное подмножество $\{x_0\} \subset X$ является ретрактом. Достаточно лишь задать отображение $r\st x \mapsto x_0$, которое, как известно, является непрерывным и при этом удовлетворяет условию $r(x_0) = x_0$.

    Однако это отчасти и является очень широким свойством, ведь ничего не говорится о том, каким образом будут происходить эти отображения. 

    Деформационный ретракт же четко задает простоту и адекватность этих деформаций пространства условием гомотопности $\mathrm{id}$.

    К примеру, нетрудно построить сильный деформационный ретракт точки n-мерного диска $r\st \overline{D^n} \to \{\mathrm{pt}\}$. Для этого необходимо предложить гомотопию между $r\st x\mapsto \mathrm{pt}$ и $\mathrm{id}\st \mathrm{pt} \mapsto \mathrm{pt} \in \overline{D^n}$. А это ровно гомотопия, предложенная при исследовании стягиваемости пространств.
\end{Ex}

Следующее предложение предлагает некоторую роль ретракта в науке.
\begin{Prop}
   Пусть подмножество $A\subset X$ является ретрактом. Тогда всякое непрерывное отображение $A\to Y$ в произвольное пространство $Y$ можно продолжить до непрерывного отображения $X\to Y$.
\end{Prop}
\begin{Proof}
    Если есть ретракт $r\st X\to A$, то для произвольного отображения  $f\st A \to Y$ можно предложить отображение $g = f \circ r \st X \to Y$. Оно будет непрерывным как композиция двух непрерывных.
\end{Proof}

А вот деформационный ретракт, как и было обещано, дает сразу гомотопическую эквивалентность.
\begin{Prop}
    Если $A\subset X$ --- деформационный ретракт $X$, то пространства $A $ и $X$ гомотопически эквивалентны.
\end{Prop}
\begin{Proof}
    Фактически в самом определении деформационного ретракта явно указана гомотопия, которая является необходимой для $i_A \circ r \sim \mathrm{id}_X$ (где $i_A$--- вложение, $r$ --- ретракт), а $r \circ i_A = \mathrm{id}_A$.
\end{Proof}


	\section{Фундаментальные группы}
\begin{Intro}
    
\end{Intro}
\subsection{Определение и свойства}
\subsection{Функториальность $\pi_1$ }
\subsection{Односвязность}


	\section{Накрытия}
\subsection{Определение и свойства}

\subsection{Поднятие отображений и лемма о поднятии}

\subsection{Фундаментальная группа окружности}

\subsection{Связь числа листов и фундаментальной группы}

\subsection{Универсальное накрытие}


	\section{Теорема ван Кампена}
\subsection{Свободное произведение групп}
\subsection{Теорема ван Кампена}
\subsection{Примеры применения теоремы}

	

\end{document}
